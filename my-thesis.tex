\documentclass[a4paper,12 pt,twoside]{report}
\usepackage{fullpage}

\usepackage{mystyle}
\pagenumbering{arabic}

\begin{document}

\begin{titlepage}

  % AUTH Logo
  \begin{minipage}{0.3\textwidth}
    \begin{flushleft}
      \includegraphics[scale=0.25]{./images/title/authLogoTr.jpg}
    \end{flushleft}
  \end{minipage}
  \begin{minipage}{0.9\textwidth}
    \begin{flushleft}
      \large Αριστοτέλειο Πανεπιστήμιο Θεσσαλονίκης \\
      Πολυτεχνική Σχολή \\
      Τμήμα Ηλεκτρολόγων Μηχανικών $\&$ \\ Μηχανικών Υπολογιστών\\
      \vspace{1mm}
      \normalsize{Τομέας Ηλεκτρονικής και Υπολογιστών} \\
      \vspace{1mm}
      \normalsize{Εργαστήριο Επεξεργασίας Πληροφορίας \\ και Υπολογισμών (IPL)} \\[5cm] 
    \end{flushleft}
  \end{minipage} \\[1.7cm]




  \begin{center}
    \Large Διπλωματική Εργασία \\[0.8cm]

    \rule{450pt}{4pt} \\[0.4cm]
    {\fontsize{20.26pt}{1em}\selectfont Σχεδίαση και Ανάπτυξη Συστήματος Ενεργής Παρακολούθησης Διαδικτυακών Εφαρμογών}

    \rule{350pt}{4pt} \\[4cm]

    % Writer
    \begin{minipage}{0.4\textwidth}
      \begin{flushleft} \normalsize
        \emph{Εκπόνηση:} \\
        Σεντονάς Σταύρος \\
        ΑΕΜ: 9386
      \end{flushleft}
    \end{minipage}
    % Supervisors
    \begin{minipage}{0.4\textwidth}
      \begin{flushright} \normalsize
        \emph{Επίβλεψη:} \\
        Καθ. Συμεωνίδης Ανδρέας\\
        Δρ. Παπαμιχαήλ Μιχαήλ \\
        Υπ. Δρ. Καρανικιώτης Θωμάς \\
      \end{flushright}
    \end{minipage}
    \\[1cm]
    \vfill

    % Title
    \large Θεσσαλονίκη, Μάρτιος 2024

  \end{center}
\end{titlepage}


\newevenside

\begin{center}
  \centering

  \vspace{0.5cm}
  \centering
  \textbf{\Large{Περίληψη}}
  \phantomsection
  \addcontentsline{toc}{section}{Περίληψη}

  \vspace{1cm}

\end{center}

  Η εξέλιξη της τεχνολογίας και της πληθώρας εφαρμογών που αναπτύσσονται στα πλαίσιο αυτής, καθιστούν επιτακτική την ανάγκη ύπαρξης συστημάτων που θα ελέγχουν την εύρυθμη λειτουργία τους. Πιο συγκεκριμένα μιλάμε για την ελέξιλη στο χώρο του διαδικτύου και των δομών που έχουν υλοποιηθεί πάνω σε αυτό.
  
  Πλέον αναφερόμαστε σε ένα συνεχώς αυξανόμενο και ευρύ δίκτυο web εφαρμογών - λογισμικών ως υπηρεσίας (SaaS - Software as a Service) που ζουν στον Διαδίκτυο (World Wide Web). H λειτουργία αυτών μπορεί να ελεχθεί με διάφορους τρόπους. Από Unit Testing, στο πλαίσιο του κύκλου ανάντυξης του λογισμικού (continuous integration, continuous deployment cycle) προκειμένου να ελεχθεί λειτουργικά το σύστημα για την αποφυγή bugs, μέχρι και Παρακολούθηση Δικτύου (Network Monitoring), για να επιβεβαιωθεί η σωστή λειτουργία των συστημάτων καθόλη της διάρκεια του κύκλου ζωής τους.
  
  Η παρούσα διπλωματική εστιάζει στην ανάπτυξη ενός συστήματος Παρακολούθησης Δικτύου και κατεπέκταση εφαρμογής που θα δίνει της δυνατότητα στους χρήστες της να παρακολουθούν, εύκολα, την ομαλή λειτουργία των διαδικτυακών σελιδών τους, είτε αυτά είναι εφαρμογές, είτε απλά στατικές σελίδες. Το σύστημα στηρίζεται στη βασική μέθοδο εντοπισμού διαθεσιμότητας μίας ιστοσελίδας, γνωστή και ως ping. Κάνοντας ping μπορούμε να πάρουμε χρήσιμη πληροφορία σχετικά με το αν το υπό μελέτη σύστημα μπορεί να ανταποκριθεί και εφόσον ανταποκριθεί σχετικά με το χρόνο που μεσολάβησε μέχρι να απαντήσει. Συνεχίζοντας την λογική πορεία ενός τέτοιου συστήματος μπορούμε ακόμα στο μήνυμα που στέλνουμε να έχουμε πληροφορία που θα επηρεάζει την απάντηση που θα περιμέναμε να δούμε, έχοντας έτσι έναν ακόμα μηχανισμό για την αναγνώριση και αποφυγή πιθανών bugs, ή λαθών κατά τη διαδικασία ανάππτυξης λογισμικών ως υπηρεσία.

\newevenside
{\fontfamily{cmr}\selectfont

\phantomsection
\addcontentsline{toc}{section}{Abstract}


\begin{center}
  \centering
  \textbf{\Large{Title}}
  \vspace{0.5cm}

  \textbf{\large{Development of a System for Uptime Status Monitoring}}
  \vspace{1cm}

  \centering
  \textbf{Abstract}
\end{center}

The evolution of technology and the abundance of applications developed within this framework make it imperative to have systems that will control their smooth operation. More specifically, we are talking about monitoring in the realm of the internet and the structures implemented on it.

Today we refer to a constantly growing and extensive network of web applications - software as a service (SaaS) - that reside on the Internet (World Wide Web), whose operations can be monitored in various ways. From unit testing, within the software development cycle (continuous integration, continuous deployment) to ensure that the system functions are running properly and avoid bugs, to network monitoring, to verify the correct functioning of the systems throughout their life cycle.

This thesis focuses on the development of a Network Monitoring system and a web application that will enable its users to easily monitor how their systems operate, whether they are applications or simply static pages. The system is based on the basic method of checking the availability of a website, known as ping. By pinging, we can obtain useful information regarding whether the system under study can respond and, if so, the time it takes to respond. Continuing the logical progression of such a system, we can further enhance the message we send with predetermined data to check the response of the system and verify the returned data, thereby identifying and avoiding potential bugs or error during the software's development process

\begin{flushright}
  \vspace{2cm}
  Sentonas Stavros
  \\
  Electrical \& Computer Engineering Department,
  \\
  Aristotle University of Thessaloniki, Greece
  \\
  June 2023
\end{flushright}

}

\newevenside

\renewcommand*\contentsname{Περιεχόμενα}
\setcounter{tocdepth}{2}

\tableofcontents

\renewcommand{\listfigurename}{Κατάλογος Σχημάτων}
\listoffigures

\renewcommand{\listtablename}{Κατάλογος Πινάκων}
\listoftables

% \listoftables
% \listofalgorithms

%\listofequations
%\newlistof{myequations}{equ}{\listequationsname}
%\listofmyequations

\chapter*{Ακρωνύμια Εγγράφου}
\label{append:acronyms}
\phantomsection
\addcontentsline{toc}{section}{Ακρωνύμια}

Παρακάτω παρατίθενται ορισμένα από τα πιο συχνά χρησιμοποιούμενα ακρωνύμια της
παρούσας διπλωματικής εργασίας:

\begin{table}[htpb]
  \centering
  \begin{tabular}{l@{$\;\;\longrightarrow\;\;$}l}
  RUM & Real User Monitoring \\
  API & Application Programming Interface \\
  SaaS & Software as a Service \\
  HTTP & Hypertext Transfer Protocol \\
  CRUD & Create, Read, Update, Delee \\
  GCS & Google Cloud Storage \\
  CPU & Central Processing Unit \\
  RAM & Random Access Memory \\
  RL & Reinforcement Learning \\
  CNN & Convolutional Neural Network \\
  LSTM & Long Short-Term Memory \\
  \end{tabular}
\end{table}



\newevenside

% %  chapter 1 = Θεώρηση Προβλήματος
% Aναφορά σε Active, Passive, Mixed Uptime monitoring
% Κίνητρο - Χρηστικότητα

\chapter{Εισαγωγή}
\label{chapter:intro}

Τα τελευταία χρόνια, ο κλάδος του Διαδικτύου προσεγγίζει ένα μεγαλύτερο κομμάτι ανθρώπων, τόσο από τη μεριά του καταναλωτή 
όσο και από τη μεριά του παραγωγού. Όσο αφορά τον καταναλωτή οι δυνατότητες που του προσφέρονται μπορούν να διακριθούν στους εξής τομέις:

\begin{itemize}
	\item Επικοινωνία: το διαδίκτυο παρέχει τη δυνατότητα άμεσης επικοινωνίας μεταξύ μεγάλων αποστάσεων, που δεν περιορίζεται μόνο στο ακουστικό
			ερέθισμα, αλλά επιτρέπει και την μετάδοση οπτικο-ακουστικής πληροφορίας
	\item Πρόσβαση Πληροφορίας: ίσως το σημεντικότερο αγαθό που προσφέρει το διαδίκτυο είναι η πληθώρα πληροφορίας
			που στεγάζει. Μηχανές Αναζήτηση (search engines), Online Βάσεις Δεδομένων (online databases),
			και άλλου είδους εφαρμογών εκπαιδευτικού χαρακτήρα που δίνουν πρόσβαση σε άτομα που το επιθυμούν, να κάνουν έρευνα
	\item Ποιότητα ζωής: σε αυτή την κατηγορία περιλαμβάνονται όλες εκείνες οι υπηρεσίες που
			διευκολύνουν την καθημερινότητα των χρηστών. Online αγορές (eshops) που γλιτώνουν την αναμονή σε ουρές ή ακόμα
			επιτρέπουν την εύκολη αγορά προϊόντων από απομακρυσμένες περιοχές του πλανήτη, ψυχαγωγία και πρόσβαση σε
			υπηρεσίες που επιταχύνουν ενέργειες που υπό άλλες περιπτώσεις θα ήταν χρονοβόρες (online banking,
			πληρωμή λογαριασμών, κρατήσεις ξενοδοχείων/εισητηρίων)
\end{itemize}

Από τη μεριά του παραγωγού, τα μέσα που υπάρχουν για την ανάπτυξη τέτοιων εφαρμογών/υπηρεσιών/συστημάτων
μέρα με τη μέρα αυξάνονται. Η ραγδαία εξέλιξη στον χώρο των cloud υποδομών, καθιστά ευκολότερη και επισπεύδει
τόσο την δημιουργία διαδικτυακών εφαρμογών, και σελιδών σε ένα γενικότερο πλαίσιο, όσο και την μεγέθυνση και αύξηση αυτών (scale up).
Μάλιστα η επιλογή κατάλληλου παρόχου τέτοιων υπηρεσιών αποτελεί ένα αρκετά σημεντικό αντικείμενο μελέτης \cite{cloud_service_provider_evaluation}.
Πέρα από τον οικονομικό παράγοντα θα πρέπει να προσμετρηθούν οι παροχές, τα πλεονεκτήματα αλλά και η αποδοτικότητα που κάθε ένας προσφέρει.

Βλέποντας λοιπόν το πόσο συνυφασμένη είναι η ζωή του σύγχρονου ανθρώπου με το δίκτυο αλλά και τις δυνατότητες και τα μέσα
που έχει ο καθένας για να αναπτύξει εφαρμογές σε αυτό, καθίσταται επιτακτική η ανάγκη ύπαρξης μηχανισμών 
που θα αναγνωρίζουν σφάλματα (bugs), θα επιβλέπουν την ορθή λειτουργία των υπό μελέτη συστημάτων καθόλη τη διάρκεια ζωής τους και θα διαθέτουν τη δυνατότητα αυτόματης εφαρμογής κατάλληλων ρυθμίσεων για τη βελτιστοποίησή τους 

\section{Περιγραφή του Προβλήματος}
\label{section:problem_description}

Η Παρακολούθηση (Monitoring) ενός συστήματος που "ζει" στο χώρο του διαδικτύου μπορεί να γίνει κυρίως με δύο τρόπους:

\begin{itemize}
	\item \textbf{Ενεργή Παρακολούθηση (Active Monitoring)}: έχει περισσότερο προγνωστικό και προληπτικό χαρακτήρα.
		Συχνά αναφέρεται και ως \textbf{Συνθετική παρακολούθηση (Synthetic Monitoring)}, λόγω της φύσης των ενεργειών της.
		Ουσιαστικά δημιουργεί πλασματικά api calls και όχι πραγματικά δεδομένα χρηστών 
		προκειμένου να ελεγχθεί η απόκριση του υπό μελέτη συστήματος. Η συχνότητα αποστολής των
		συνθετικών αιτημάτων συνήθως ρυθμίζεται από το χρήστη.
	\item \textbf{Παθητική Παρακολούθηση (Passive Monitoring)}: παρέχει μία πιο πλήρη εικόνα σχετικά με πως χρησιμοποιούνται οι πόροι του δικτύου
		καταγράφοντας, αποθηκεύοντας και αναλύοντας τα δεδομένα του χρήστη. Για αυτό πολλές φορές αναφέρεται στη βιβλιογραφία ως \textbf{Παρακολούθηση Πραγματικών Χρηστών (Real User Monitoring - RUM)}. Έτσι μπορεί κανείς
		να εντοπίσει τις τάσεις χρήσης του δικτύου για τη βελτίωση και βελτιστοποίησή του συστήματος.
\end{itemize}


\begin{table}[H]
	\begin{center}
		\caption{Χαρακτηριστικά Ενεργής και Παθητικής Παρακολούθησης}
		\label{tab:active_vs_passive_monitoring}
		\begin{tabular}{ | c | c | }
			\hline
				\thead{Ενεργή Παρακολούθση \\ (Active Monitoring)} & \thead{Παθητική Παρακολούθηση \\ (Passive Monitoring)} \\
			\hline
				% \makecell{$\bullet$ Παράγει μικρή ποσότητα \\ δεδομένων} & \makecell{$\bullet$ Παράγει μεγάλη ποσότητα \\ δεδομένων} \\
				\makecell{$\bullet$ Στηρίζεται σε συνθετικά \\ API calls} & \makecell{$\bullet$ Αναλύει δεδομένα πραγματικών \\ χρηστών} \\
				\makecell{$\bullet$ Παράγει δεδομένα για συγκεκριμένες \\ πτυχές του δικτύου} & \makecell{$\bullet$ Πλήρης εικόνα της απόδοσης \\ του δικτύου} \\
				\makecell{$\bullet$ Μπορεί να μετρήσει την κίνηση \\ εντός και εκτός του δικτύου} & \makecell{$\bullet$ Μετράει κίνηση μόνο \\ εντός του δικτύου} \\
				\makecell{$\bullet$ Μπορεί να εντοπίσει προβλήματα \\ πριν ακόμα μπορέσουν να τα \\ εντοπίσουν οι χρήστες} & \makecell{$\bullet$ Εντοπίζει προβλήματα που \\ εμφανίζονται εκείνη τη στιγμή} \\
			\hline
		\end{tabular}
	\end{center}
\end{table}

Και οι δύο μέθοδοι έχουν πλεονεκτήματα και μειονεκτήματα, τα οποία φαίνονται και στον παραπάνω πίνακα \ref{tab:active_vs_passive_monitoring}. Όπως είναι εμφανές
η παθητική παρακολούθηση γίνεται πάνω στο σύστημα που θέλουμε να μελετήσουμε, πράγμα το οποίο σήμαινει ότι σαν εξωτερικοί παράγοντες στο
σύστημα δεν θα μπορέσουμε να προσφέρουμε ανάλογες υπηρεσίες. Για το λόγο αυτό συνεχίζουμε την ανάλυση στο πλαίσιο της Ενεργής Παρακολούθησης Δικτύων.

Οι βασικοί λόγοι που χρειάζονται τέτοιου είδους υπηρεσίες όπως αναφέρεται και στα \cite{web_server_monitoring}, \cite{synthetic_monitoring_using_http_archive}
είναι οι εξής:

\begin{itemize}
	\item Βελτίωση προβλημάτων που σχετίζονται με την απόδοση του συστήματος
		πρωτού τα βιώσουν οι πραγματικοί χρήστες του συστήματος
	\item Ύπαρξη κάποιας μονάδας αξιολόγησης της απόδοσης του
	\item Αξιολόγηση του συστήματος υπό μεγαλύτερο φορτίο
	\item Διασφάλιση της Συμφωνίας Επιπέδου Υπηρεσιών (Service Level Agreement - SLA), μεταξύ
		του παρόχου υπηρεσιών και των χρηστών
	\item Παρέχει χρήσιμα δεδομένα ακόμα και σε καινούργια συστήματα που ακόμα μπορεί να μην έχουν χρήστες   
\end{itemize}

\section{Σκοπός - Συνεισφορά της Διπλωματικής Εργασίας}
\label{section:contribution}

Η παρούσα διπλωματική εργασία μελετά τη χρήση σύγχρονων τεχνολογιών για τη δημιουργία
ενός συστήματος Ενεργής Παρακολούθησης (Active Monitoring) σε συνδυασμό με μία SaaS εφαρμογή
που θα παρουσιάζει μέσα από διαγράμματα τα αποτελέσματα της ανάλυσης της πληροφορίας που εξάγεται.

Εξετάζονται διάφοροι τρόποι και υλοποιήσεις που δοκιμάστηκαν κατά τη διάρκεια
εκπόνησεις της διπλωματικής αυτής εργασία, και τέλος θα αναλύσουμε τα αποτελέσματα
που παράξαμε καθόλη της διάρκεια των πειραμάτων που διενεργήθηκαν.  
\section{Διάρθρωση της Αναφοράς}
\label{section:layout}

Η διάρθρωση της παρούσας διπλωματικής εργασίας είναι η εξής:

\begin{itemize}
  \item{\textbf{Κεφάλαιο \ref{chapter:theory}:} 
		Περιγράφονται τα βασικά εργαλεία και θεωρητικά στοιχεία
		στα οποία βασίστηκαν οι υλοποιήσεις
    }
  \item{\textbf{Κεφάλαιο \ref{chapter:state_of_the_art}} Αναφορά συστημάτων που ήδη χρησιμοποιούνται	
		και παράθεση διαφορών με την υλοποίησή μας
    }
  \item{\textbf{Κεφάλαιο \ref{chapter:implementations}} Περιγραφή των υλοποιήσεων
  		και πλήρης περιγραφή του τελικού συστήματος
    }
  \item{\textbf{Κεφάλαιο \ref{chapter:system_showcase}} Παρουσιάζονται τα αποτελέσματα μετρήσεων
    που έγιναν στο τελικό σύστημα καθώς και εικόνες της γραφικής διεπαφής που υλοποιήσαμε.
    }
  \item{\textbf{Κεφάλαιο \ref{chapter:end}} Προτείνονται θέματα για μελλοντική
      μελέτη, αλλαγές και επεκτάσεις.
    }
\end{itemize}


\newevenside
% \newevenside

% % chapter 2 = Θεωρητικό Υπόβαθρο
% - HTTP ======== DONE
% - API/ REST + WEBSOCKET ========== DONE
% - ?SaaS?
% - avro file format (explain what it is and why its good) ========== DONE
% - (probably not) στατιστικά που χρησιμοποιούνται (μέση τιμέ, διάμεσος, διασπορά, τεταρτημόρια) ========== DONE
% - ******εργαλεία******
% - node server (?express?) + pm2
% - dbs -> sql vs nosql and why nosql is better in this system ======= LEFT OUT
% - object storage (google cloud storage) ======= LEFT OUT
% - Socket.io (used in one of the implementations, could also be added in the future instead of only rest api's) -> could be mentioned in tools (its an abstract layer over websockets)

\chapter{Θεωρητικό Υπόβαθρο}
\label{chapter:theory}

Στο κεφάλαιο αυτό θα παρουσιαστούν εργαλεία που χρησιμοποιήθηκαν για την υλοποίηση του Συστήματος
Ενεργής Παρακολούθησης, καθώς και έννοιες και τεχνολογίες που αξιοποιήθηκαν για το σκοπό αυτό.

\section{Hypertext Transfer Protocol}
\label{section:http}

Το πρωτόκολλο επικοινωνίας HTTP (Hypertext Transfer Protocol) αποτελεί το πιο διαδεδομένο και ευρέως γνωστό
πρωτόκολλο στο χώρο του διαδικτύου. Αναπτύχθηκε από τους Tim Berners-Lee και την ομάδα του το 1990 και από τότε
έχει περάσει πολλές αλλαγές προκειμένου να μπορεί να ανταπεξέλθει στις ολοένα και συνεχώς αυξανόμενες ανάγκες του σήμερα.

Αποτελεί τη βάση κάθε μετάδοσης πληροφορίας στο διαδίκτυο. Στηρίζεται στην επικοινωνία δύο υπολογιστών, ενός που κάνει τα αιτήματα (client)
και ενός που απαντά σε αυτά (server). Στο τέλος της επικοινωνίας στην μεριά του παραλήπτη θα υπάρχει ανακατασκευασμένο
ένα ολοκληρωμένο αρχείο, από τα διάφορα υπο-αρχεία που μαζεύτηκαν, που μπορεί να είναι αρχεία ήχου, εικόνας, video.
Τα αιτήματα αυτού που ξεκινάει την επικοινωνία ονομάζονται requests, ενώ οι απαντήσεις του αποστολέα responses.

Η βασική δομή ενός http αιτήματος, η οποία φαίνεται και στο \autoref{fig:http_request}, περιληπτικά περιλαμβάνει τη μέθοδο (method) του αιτήματος, που περιγράφει
τη βασική λειτουργία του, το μονοπάτι (path) στο οποίο θα επικοινωνήσει με τον server, την έκδοση του πρωτοκόλλου που 
θα χρησιμοποιηθεί και τέλος headers προκειμένου να κρίνει ο server αν πρέπει να απαντήσει ή όχι πίσω στον client

\begin{figure}[!ht]
	\centering
	\includegraphics[width=0.7\textwidth]{./images/chapter2/http_request.png}
	\caption[Βασική Δομή ενός αιτήματος http]{Βασική Δομή ενός αιτήματος http}
	\label{fig:http_request}
\end{figure}

\subsection{Μέθοδοι}
\label{subsec:http_methods}

Πιο συγκεκριμένα οι βασικές μέθοδοι που παρέχει το http και οι συνήθεις λειτουργίες τους είναι οι εξής:

\begin{itemize}
	\item \textbf{GET}: παίρνει πληροφορία από τον server
	\item \textbf{POST}: υποβάλλει πληροφορία, προκαλώντας αλλαγές στον τρόπο λειτουργίας του server. Σχετίζεται συχνά με τη δημιουργία πληροφορίας που προηγουμένως δεν υπήρχε 
	\item \textbf{PUT}: όπως και πριν στέλνει πληροφορία στον παραλήπτη υπολογιστή, αλλά αυτή τη φορά επηρεάζει πόρους που ήδη υπήρχαν στο σύστημα. Σχετίζεται συχνά με την τροποποίηση ήδη υπάρχουσας πληροφορίας
	\item \textbf{DELETE}: διαγράφει από το σύστημα του server το συγκεκριμένο πόρο.
\end{itemize}

Αξίζει να σημειωθεί ότι πέρα από τις τέσσερις αυτές βασικές μεθόδους υπάρχουν και άλλες όπως είναι 
η \textbf{PATCH} που αποτελεί ειδική περίπτωση της PUT, η \textbf{HEAD} που αποτελεί ειδική περίπτωση της GET,
καθώς και άλλες που σχετίζονται με τη σύνδεση μεταξύ server και client. Αυτές είναι οι \textbf{CONNECT}, \textbf{OPTIONS} και \textbf{TRACE}.
Καθώς όμως, οι υπόλοιπες αυτές οι μέθοδοι, δεν χρησιμοποιούνται τόσο συχνά στην πράξη, δεν θα αναλυθούν περαιτέρω.

\subsection{Εκδόσεις HTTP}
\label{subsec:http_versions}

Η πρώτη έκδοση του HTTP, παρόλλο που δεν είχε κάποια συγκεκριμένo τίτλο, εκ των υστέρων ονομάστηκε 
HTTP/0.9. Αποτελεί την πιο απλή έκδοση του πρωτοκόλλου. Δεν υποστηρίζονταν headers και κωδικοί κατάστασης (status codes).
Εξυπηρετούσε μόνο GET αιτήματα και η μοναδική απάντηση που μπορούσε να επιστρέψει ήταν hypertext αρχεία. Kάθε φορά που ο server ανταποκρινόταν και έστελνε
απάντηση, η επικοινωνία με τον client έκλεινε κατευθείαν.

Στη συνέχεια και με την ανάπτυξη του διαδικτύου προστέθηκαν και άλλες λειτουργίες. Πέριξ του 1996,
με τη επόμενη έκδοση του πρωτοκόλλου (HTTP/1.0) τα αιτήματα πλέον συνοδεύονταν από headers, μεταπληροφορία σχετικά
με τη κατάσταση του αιτήματος, τον τύπο της πληροφορίας που περιμένουμε να έρθει (stylesheets, media, hypertext) καθώς και 
την έκδοση του HTTP που χρησιμοποιήθηκε στη συγκεκριμένη επικοινωνία. Επιπλέον πέρα από τη GET μέθοδο υπάρχει η
δυνατότητα για POST και PUT, δημιουργία και τροποποίηση πληροφορίας δηλαδή.

Στη συνέχεια το HTTP/1.1 προσπαθεί να βελτιώσει τις ήδη υπαρχουσες δυνατότητες κάνοντας την επικοινωνία
μεταξύ server και client πιο αποδοτική. Αντί να κλείνει η επικοινωνία μετά από κάθε μήνυμα, η σύνδεση παραμένει
ανοιχτή γλιτώνοντας έτσι μία σταθερή καθυστέρηση που υπήρχε σε κάθε αίτημα

Φτάνοντας στο σήμερα, μιλάμε για το HTTP/2.0 \cite{http2}. Αξιοποιώντας το πρωτόκολλο Speedy (SPDY) που αναπτύχθηκε κάποια χρόνια πριν
την κυκλοφορία του, και κτίζοντας πάνω σε αυτό, κατάφερε να μειώσει τους χρόνους επικοινωνίας server-client.
Μερικοί από τους τρόπους που επιτυγχάνεται αυτό είναι η μετατροπή του http από text πρωτόκολλο, σε δυαδικό (binary protocoll), επιτρέποντας έτσι χρήση καλύτερων
και αποδοτικότερων τεχνικών επικοινωνίας. Επιπλέον συμπιέζει τους headers (header compression) καθώς αποτελούν πληροφορία που
επαναλαμβάνεται όταν τα αιτήματα στον server είναι συνεχή. Ο server ακόμα, αποκτά έναν μηχανισμό (server-push) που του
επιτρέπει να προωθεί πληροφορία στον client (στην cache του client συγκεκριμένα), που δεν έχει ζητήσει ακόμα, αλλά βάση αυτού
που αιτήται, μάλλον θα ζητήσει εντός της ιδίας συνεδρίας.

Τέλος, πρέπει να αναφερθούμε στην τελευταία, αν και όχι ακόμα ευρέως διαδεδομένη, έκδοση HTTP/3.0. H βασική διαφορά με τους
πρωκατόχους του είναι ότι αλλάζει το πρωτόκολλο επικοινωνίας που χρησιμοποιεί όλα αυτά τα χρόνια, από TCP (Transfer Communication Protocoll) σε
έναν συνδυασμό UDP (User Datagram Protocoll) και QUIC (Quick UDP Internet Connections), μίας νέας τεχνολογίας που λύνει το πρόβλημα και βελτιστοποιεί τόσο το πρόβλημα 
της ασφάλειας των επικοινωνιών (TLS handshakes), όσο και της απώλειας πληροφορίας που μπορεί να υπήρχε λόγω UDP, πρωτοκόλλου που είναι γνωστό
για την ταχύτερη απόδοσή του σε σχέση με το TCP, αλλά και το γεγονός ότι είναι πιο επιρρεπές σε σφάλματα. Η νέα αυτή έκδοση από τα αποτελέσματα 
του \cite{http3} φαίνεται να έχει ήδη καλύτερους χρόνους σε σχέση με τις παλαιότερες εκδόσεις και ήδη το 28\% του διαδικτύου αξιοποιεί τις δυνατότητές του. 


\subsection{Κωδικοί Κατάστασης}
\label{subsec:http_status_codes}

Οι κωδικοί κατάστασεις (status codes) αποτελούν μέρος της απάντησης του server. Επιτρέπουν στον χρήστη να καταλάβει με μία ματιά αν το αίτημα που έχει κάνει έχει επιστρέψει σωστά, ή έχει γίνει κάποιο λάθος στη μεριά του server.
Υπάρχουν πέντε μεγαλύτερες κατηγορίες που στεγάζουν όλες τις υποπεριπτώσεις αυτών. Πιο συγκεκριμένα:

\begin{itemize}
	\item \textbf{Εύρος 100-199}: Υποδηλώνουν ενημερωτική απάντηση σχετικά με τη λειτουργία του server
	\item \textbf{Εύρος 200-299}: Επιτυχή αιτήματα. 
	\item \textbf{Εύρος 300-399}: Υποδηλώνουν την ανακατεύθυνση του μηνύματος του client. Συνήθως συνοδεύονται από το νέο url στο οποίο πρέπει να αποσταλλεί το αίτημα
	\item \textbf{Εύρος 400-499}: Ανεπιτυχές αίτημα, που οφείλεται στον client. Ένα σύνηθες παράδειγμα είναι η αίτηση πρόσβασης σε προστατευόμενους πόρους χωρίς κάποιου είδους αυθεντικοποίηση, ή χωρίς τα σωστά στοιχεία για αυθεντικοποίηση
	\item \textbf{Εύρος 500-599}: Ανεπιτυχές αίτημα, που οφείλεται στον server. 
\end{itemize}
% \input{./chapters/chapter3_theory/section1_dnn.tex}
% \input{./chapters/chapter3_theory/section2_cnn.tex}
% \input{./chapters/chapter3_theory/section3_sota.tex}
\newevenside

% % chapter 3 = State of the art
% - αναφορά σε commercial και open source projects
% - παράθεση πλεονεκτημάτων της δικής μου υλοποίησης (βασικά εδώ αναφέρουμε τα: 
% (1)infinite scalability και ίσως το γεγονός ότι μπορούμε να (
% (2) κοιτάμε σε μεγάλο βάθος χρόνου για να παρουσιάσουμε "συνολικά στατιστικά", χωρίς αυτό να επιβαρύνει τη βάση μας ή να δυσχαιρένει τη λειτουργία του συστήματος μας). 
% Το γεγονός ότι έχει infinite scalebility μας παρέχει τη δυνατότητα να έχουμε
% (3)όσα 1m intervals θέλουμε και ακόμα τη δυνατότητα να κάνουμε stress test με όσα requests per minute θέλει ο χρήστης.
% Tέλος τα περισσότερα ενώ κάποια από αυτά τασ εργαλεία σου δίνουν τη δυνατότητα να κάνεις customise τα μηνυματα που στέλνεις,
% τα περισσότερα δε σε αφήνουν να τροποποιήσεις τα apis σου.

% commercial products - Better Uptime, Pulsetic, Datadog, Freshping, Hyperping, UptimeRobot
% open souce - Upptime, Uptime Kuma, Cabot, Zabbix, Sensu

\chapter{Βιβλιογραφική Αναζήτηση Τεχνολογιών Αιχμής}
\label{chapter:state_of_the_art}

Πριν αναλυθεί η υλοποίηση του συστήματος που δημιουργήσαμε για την Ενεργή Παρακολούθηση
Eφαρμογών ως Υπηρεσίες (SaaS) και ιστοσελιδών που εδρεύουν στο διαδίκτυο, θα πασουσιάσουμε τεχνολογίες
που έχουν χτιστεί ήδη για τον σκοπό αυτό και θα δείξουμε τους τομείς στους οποίους
διαφέρει το δικό μας σύστημα.

Εφαρμογές τέτοιου τύπου έχουν αναπτυχθεί κυρίως από εταιρίες, αλλά υπάρχουν πολλά open source
projects μικρότερου βεληνεκούς που επιτυγχάνουν τον ίδιο στόχο. Στη συνέχεια θα αναφερθούμε κυρίως στα
πιο γνωστά και διαδεδομένα εργαλεία, παρουσιάζοντας τα δυνατά τους σημεία και περιγράφοντας τις λειτουργίες
που παρέχουν.

\begin{itemize}
	\item \textbf{Better Uptime}: Προσφέρει εύκολη ενσωμάτωση των ιστοσελιδών που θέλει κανείς να παρακολουθήσει.
	      Λειτουργεί κάνοντας ping κάθε τριάντα δευτερόλεπτα στο url που ορίζει ο χρήστης και παρουσιάζει
	      τα παραγόμενα δεδομένα σε ευπαρουσίαστα διαγράμματα (\autoref{fig:better_uptime}). Ένα από τα μεγάλα πλεονεκτήματα που έχει αφορά
	      τη δυνατότητα για πολλαπλά ping από διαφορετικές περιοχές του κόσμου (Ευρώπη, Ασία, Βόρεια Αμερική, Αυστραλία),
	      ώστε οι χρήστες να διαθέτουν μία πιο πλήρη εποπτεία του υπό μελέτη συστήματος/ιστοσελίδας. Αξίζει να σημειωθεί
	      ότι παρέχει και μηχανισμούς ενημέρωσης για να ειδοποιεί το χρήστη σε περίπτωση μη απόκρισης τους συστήματος, μέσα
	      από mail, εφαρμογές chatting και τηλεφωνικών κλήσεων.
	      \begin{figure}[!ht]
		      \centering
		      \includegraphics[width=0.7\textwidth]{./images/chapter3/better-uptime-cropped.png}
		      \caption[Παράδειγμα χρήσης του εργαλείου Better Uptime]{Παράδειγμα χρήσης του εργαλείου Better Uptime}
		      \label{fig:better_uptime}
	      \end{figure}
	\item \textbf{Uptime Robot}: Επιτρέπει, πέρα από την επιλογή του url, και την επιλογή παραπάνων παραμέτρων που επηρεάζουν
	      την απάντηση που θα επιστρέψει το υπό μελέτη σύστημα. Οι παράμετροι αυτοί σχετίζονται με τους headers του μηνύματος που
	      αποστέλλεται, και πιο συγκεκριμένα, με αυτούς που αφορούν την αυθεντικοποίηση του χρήστη (στην προκειμένη περίπτωση
	      του συστήματος παρακολούθησης). Πέρα από αυτά μπορεί να καθορίσει την επιθυμητή http κατάσταση της απόκρισης
	      της ιστόσελίδας και το χρόνο που θα παρεμβάλλεται μεταξύ διαδοχικών αιτημάτων (pings). Τέλος, διαθέτει κάποια βασικά διαγράμματα
	      που σχετίζονται με το αν η απόκριση του συστήματος είναι ορθή ή όχι (\autoref{fig:uptime_robot}).
	      \begin{figure}[!ht]
		      \centering
		      \includegraphics[width=0.7\textwidth]{./images/chapter3/uptime_robot.png}
		      \caption[Παράδειγμα χρήσης του εργαλείου Uptime Robot]{Παράδειγμα χρήσης του εργαλείου Uptime Robot}
		      \label{fig:uptime_robot}
	      \end{figure}
	\item \textbf{Site24x7}: Διαθέτει μετρικές, που αφορούν τη μέγιστη/ελάχιστη τιμή του χρόνου απόκρισης του συστήματος, καθώς και τη μέση τιμή του,
	      ενώ παράλληλα δίνει μία εικόνα του throughput του συστήματος. Δεν λείπει φυσικά και ένα διάγραμμα απόκρισης χρόνου (\autoref{fig:site24x7})
	      που καθιστα τα δεδομένα που συλλέγονται πιο εύκολα στην κατανόηση και οπτικοποίηση.
	      \begin{figure}[!ht]
		      \centering
		      \includegraphics[width=0.7\textwidth]{./images/chapter3/site24x7.png}
		      \caption[Παράδειγμα χρήσης του εργαλείου Site24x7]{Παράδειγμα χρήσης του εργαλείου Site24x7}
		      \label{fig:site24x7}
	      \end{figure}
	\item \textbf{Uptimia}: Πέρα από τα κλασικού τύπου http αιτήματα, μπορεί να κάνει ελέγχους παρακολούθησης (uptime monitoring)
	      σε DNS, UDP, TCP και email με απόσταση έως και τριάντα δευτερολέπτων (μεταξύ αιτημάτων). Αξίζει, να σημειωθεί, ότι η συγκεκριμένη
	      εφαρμογή έχει δυνατότες και παθητικής παρακολούθησης (RUM). Αρχικά επιλέγεται το site το οποίο ο χρήστης θέλει να παρακολουθήσει, καθώς και τα δεδομένα για το οποία επιθυμεί να ενημερώνεται ή να παρακολουθεί.
		  Αυτά σχετίζονται, κυρίως με σφάλματα ή καταστάσεις στις οποίες βρίσκεται το σύστημα και μπορεί να δηλώνουν κάποιο πρόβλημα.
		  Καταστάσεις όπως είναι η μειωμένη απόδοση του υπό μελέτη συστήματος ή η απότομη πτώση του πλήθους των χρηστών μίας σελίδας.
	      Για να επιτύχει τέτοιας μορφής ελέγχους, παράγει (ανάλογα με τον τύπο των ελέγχων που επιλέγουμε)
	      ένα script γραμμένο σε JavaScript που τοποθείται στην αρχή της ιστοσελίδας την οποία θέλουμε να παρακολουθήσουμε.
	      Με αυτό τον τρόπο δίνεται η δυνατότητα συλλογής δεδομένων πραγματικών χρηστών.
	      \begin{figure}[!ht]
		      \centering
		      \includegraphics[width=0.7\textwidth]{./images/chapter3/uptimia.png}
		      \caption[Παράδειγμα χρήσης του εργαλείου Uptimia]{Παράδειγμα χρήσης του εργαλείου Uptimia}
		      \label{fig:uptimia}
	      \end{figure}
\end{itemize}

\break

Παραπάνω αναφέρθηκαν μερικά μόνο κάποια από τα εργαλεία που υπάρχουν σήμερα για την Ενεργη Παρακολούθηση του Χρόνου
και Απόκρισης Ιστοτόπων και Διαδικτυακών Εφαρμογών. Σε αυτά θα πρέπει να προστεθούν πληθώρα εφαρμογών όπως τα:
\textbf{StatusCake}, \textbf{SemaText}, \textbf{Uptrends}, \textbf{Dotcom-monitor}, \textbf{Updown}, \textbf{Datadog Synthetics}.
Οι δυνατότες που προσφέρουν ως επί το πλείστον μπορούν περιγραφούν πλήρως από όσα αναπτύξαμε προηγουμένως, για αυτό το λόγω δεν θα αναφερθούμε περαιτέρω.

Πρέπει σε αυτό το σημείο όμως, να τονίσουμε ότι όσα προαναφέρθηκαν αποτελούν προϊόντα εταιριών.
Αυτό όμως δεν σταματάει την ανάπτυξη open source projects που υλοποιήθηκαν από χρήστες είτε ως προσωπικά projects, είτε
ως projects μίας μεγαλύτερης ομάδας από developers. Έτσι και στο πλαίσιο της Ενεργής Παρακολούθησης μερικά από τα πιο
γνωστά και δημοφιλή μεταξύ developers αποθετήρια αποτελούν τα:

\begin{itemize}
	\item \href{https://github.com/upptime/upptime}{Upptime\footnote{αποθετήριο κώδικα Upttime: \textit{https://github.com/upptime/upptime}}}: Αποτελεί μία ενδιαφέρουσα προσέγγιση στο πρόβλημα
	      της διαχείρησης των schedulers που θα πρέπει να έχει το σύστημα για να κάνει αιτήματα ανά ένα συγκεκριμένο
	      και σταθερό χρονικό διάστημα. Προκειμένου να επιτύχει κάτι τέτοιο αξιοποιεί τις δυνατότητες του GitHub (cloud-based υπηρεσία αποθήκευσης git αποθετηρίων)
	      και των actions που αυτό σου επιτρέπει να εκτελείς κάθε πέντε λεπτά. Έτσι λοιπόν ανά πέντε λεπτά (ελάχιστος χρόνος ελέγχου απόκρισης)
	      τρέχει αυτοματοποιημένα και μέσω του GitHub (ανεξάρτητα από το σύστημα αυτό) μία διαδικασία που κάνει αιτήματα στην σελίδα που ο χρήστης ορίζει.
	      Φυσικά τα αποτελέσματα αυτά αποθηκεύονται και ανά έξι ώρες παράγονται διαγράμματα που μπορείς να τα δεις μέσα από μία σελίδα που παράγεται αυτόματα.
	      \begin{figure}[!ht]
		      \centering
		      \includegraphics[width=0.8\textwidth]{./images/chapter3/nagios.jpg}
		      \caption[Αρχιτεκτόνική Συστήματος Nagios]{Αρχιτεκτόνική Συστήματος Nagios}
		      \label{fig:nagios}
	      \end{figure}
	\item \href{https://github.com/NagiosEnterprises/nagioscore}{Nagios\footnote{αποθετήριο κώδικα Nagios: \textit{https://github.com/NagiosEnterprises/nagioscore}}}: Είναι ένα πρόγραμμα γραμμένο στη γλώσσα προγραμματισμού
	      C. Πέρα από το backend διαθέτει και Γραφικό Περιβάλλον Χρήστη (Graphical User Interface - GUI).
	      Αποτελείται από ένα σύστημα ενός server (nagios server), o oποίος λειτουργεί σαν ένας scheduler που στέλνει σήματα
	      για να ξεκινήσει την εκτέλεση plugins στα απομακρυσμένα συστήματα που θέλουμε να παρακολουθήσουμε. Μόλις τα plugins δεχτούν
	      απάντηση επιστρέφουν στον server o οποίος προωθεί την πληροφορία στο GUI για να τη δούνε και οι χρήστες. Δεν χρησιμοποιείται κάποια βάση
	      δεδομένων, καθώς η πληροφορία που μαζεύεται αποθηκεύεται μόνο σε logs στο σύστημα που τρέχει την υπηρεσία αυτή. Την αρχιτεκτονική του συστήματος μπορούμε
		  δούμε στo \autoref{fig:nagios}.
	\item \href{https://github.com/louislam/uptime-kuma}{Kuma Uptime\footnote{αποθετήριο κώδικα Kuma Uptime: \textit{https://github.com/louislam/uptime-kuma}}}: Αποτελεί μία εύκολή, στη χρήση και στήσιμο,
	      self-hosted εφαρμογή γραμμένη σε JavaScript για Ενεργή Παρακολούθηση. Για την αποστολή των αιτημάτων σε απομακρυσμένες (στο διαδίκτυο) σελίδες
	      χρησιμοποιεί \textbf{child\_proccesses}, ένα api δήλαδή που περιλαμβάνει η Node.js για τη δημιουργία διεργασιών (processes) εντός άλλων διεργασιών.
	      Η εφαρμογή περιλαμβάνει backend και frontend, στο οποίο εμφανίζονται τα αποτελέσματα του monitoring. Αξίζει να σημειωθεί
	      ότι τα δεδομένα αποθηκεύονται σε βάση SQLite, ένα προσωρινό σύνολο δεδομένων που υφίσταται μόνο στο πλαίσιο εκτέλεσης μίας εφαρμογής.
	      Αποτελεί μία server-less βάση δεδομένων και είναι άρρηκτα συνδεδεμένη με την εφαρμογή στην οποία υπάρχει.
	      \begin{figure}[!ht]
		      \centering
		      \includegraphics[width=0.7\textwidth]{./images/chapter3/kuma-uptime.jpg}
		      \caption[Παράδειγμα χρήσης του εργαλείου Kuma Uptime]{Παράδειγμα χρήσης του εργαλείου Kuma Uptime}
		      \label{fig:kuma_uptime}
	      \end{figure}
\end{itemize}

Κάθε σύστημα που αναφέρθηκε έχει πλεονεκτήματα αλλά και μειονεκτήματα σε σχέση με άλλα.
Αυτό που θα θέλαμε να διαθέτει ιδανικά ένα σύστημα Παρακολούθησης, βάσει όλων αυτών που είδαμε μέχρι τώρα,
είναι τα εξής:

\begin{enumerate}
	\item Σταθερό και Αξιόπιστο σύστημα διαχείρησης προγραμματισμού (scheduling system) που θα
		καθορίζει το πότε πρέπει να ξεκινάνε τα αιτήματα προς τα εξωτερικά υπό παρακολούθηση συστήματα.
	\item Κάποια μορφή βάσης δεδομένων στην οποία θα αποθηκεύουμε τα δεδομένα που συλλέγουμε.
	\item Χρήση ιστορικών δεδομένων για τον υπολογισμό μετρικών στατιστικής φύσης σε βάθος χρόνου.
	\item Προβολή των δεδομένων σε ευπαρουσίαστα και εύπεπτα διαγράμματα που θα βοηθούν το χρήστη να αντιλαμβάνεται
		γρήγορα την κατάσταση των συστημάτων του
	\item Διάθεση τρόπων ειδοποίησης του χρήστη σε περίπτωση που κάποιο από τα συστήματα δεν ανταποκρίνεται
	\item Δυνατότητα για οριζόντια κλιμάκωση (\textbf{horizontal scaling})
	\item Ελαχιστοποίηση downtime του συστήματος 
	\item Πλήρης παραμετροποίηση του μηνύματος που αποστέλλεται (κατάλληλη ρύθμιση headers, body και query)
	\item Ρύθμιση του χρόνου μεταξύ διαδοχικών μηνυμάτων
\end{enumerate}

Πολλές απο τις εφαρμογές που είδαμε καλύπτουν σε κάποιο βαθμό μερικά από τα παραπάνω, αλλά καμία
δεν τα καλύπτει όλα ταυτόχρονα. Τα εργαλεία που αναπτύχθηκαν σε enterprise επίπεδο έχουν δυνατότητες κλιμάκωσης
αλλά στα περισσότερα (αν όχι σε όλα) μπορείς να δεις πληροφορία μέχρι ένα συγκεκριμένο χρονικό διάστημα πίσω στο χρόνο,
κρύβοντας δεδομένα που είτε δεν αποθηκεύουν πλέον είτε δεν μπορούν να αντλήσουν αρκετά γρήγορα.
Απο την άλλη οι περισσότερες open source εφαρμογές που αναφέραμε δεν μπορούν να κάνουν τόσο εύκολα scaling
καθώς είναι φτιαγμένες να δουλεύουν για περιορισμένο πλήθος χρηστών. 

Στο πλαίσιο της τρέχουσας διπλωματικής καλούμαστε να δημιουργήσουμε ένα τέτοιο σύστημα Ενεργής Παρακολούθησης που θα καλύπτει τα παραπάνω. Παράλληλα θα έχει δυνατότητες επικοινωνίας με το ίδιο το μηχάνημα στο οποίο τρέχει το υπό μελέτη σύστημα προκειμένου να λαμβάνει διαγνωστικά της λειτουργίας του (κατανάλωση CPU και υπολογιστικής μνήμης RAM) και να λαμβάνει αποφάσεις, ανάλογα με την απόκριση του συστήματος, για τη διανομή πόρων αυτού.

Όσο αφορά το τελευταίο κομμάτι, στη βιβλιογραφία, έχουμε βρει ανάλογες προσπάθειες για την βέλτιστη διανομή των υπολογιστικών πόρων εφαρμογών/ιστοσελι-δών που έχουν δημιουργηθεί με τη χρήση τεχνολογίας docker, υπό μορφή containers. Μερικές από αυτές είδαμε στα \cite{sinan}, \cite{reclaimer} όπου χρησιμοποιούνται machine learning μέθοδοι και αλγόριθμοι προκειμένου να γίνουν προβλέψεις για τη λειτουργία του συστήματος που μελετάμε και να τροποποιηθούν όπως κάθε φορά πρέπει οι αντίστοιχοι πόροι αυτού. Οι αλλαγές γίνονται προμειμένου να εξασφαλιστεί η Quality of Service (QoS) απαίτηση σχετικα με το μέγιστο latency end-to-end, που δεν πρέπει να ξεπερνάει το 99ο εκατοστημόριο (99 percentile end-to-end latency) \cite{qos_1} \cite{qos_2}. Πιο συγκεκριμένα στον Sinan \cite{reclaimer} χρησιμοποιείται Reinforcement Learning (RL) που βασίζεται σε Markov Decision Process. Ανά συγκεκριμένενες χρονικές στιγμές λαμβάνει δεδομένα από το ίδιο το docker container που ελέγχει και αποφασίζει για το πως θα διανείμει τη CPU κάθε φορά. Στο Sinan \cite{sinan} αξιοποιεί Convolutional Neural Networks (CNN) για να κάνει προβλέψεις, με δεδομένα από το διαχειριστικό του Docker, τις οποίες στη συνέχεια βάζει σαν είσοδο σε Boosted Tress. Αυτά λαμβάνουν την απόφαση για το πως θα γίνει η αναδιανομή των πόρων στα υπο μελέτη Docker containers. Οι πόροι που τροποποιεί είναι η CPU και η μνήμη RAM. Προκειμένου να λειτουργήσουν τα Boosted Trees θα πρέπει να γίνει ένα pre-training με πραγματικά δεδομένα στο υπό μελέτη σύστημα για όσες το δυνατό περισσότερες περιπτώσεις στις οποίες αυτό μπορεί να βρεθεί (διαφορετικοί συνδυασμοί CPU, μνήμης RAM, latency, traffic). 

Αξίζει να αναφερθεί ότι στο πλαίσιο βελτίωσης της λειτουργίας docker containers, πέραν της κατακόρυφης κλιμάκωσης (vertical scaling) υπάρχουν μέθοδοι οριζόντιας κλιμάκωσης (horizontal scaling) όπως αναφέρεται και στο \cite{dynamic_resource_allocation}, όπου προσπαθούν να βρουν κατάλληλο μηχάνημα κάθε φορά, για να δημιουργηθούν τα καινούργια services που θέλουν να προσθέσουν ή να αναβαθμίσουν τα ήδη υπάρχοντα. 

Στη δική μας υλοποίηση το dynamic resource allocation βασίζεται όπως θα δούμε και στη συνέχεια στην αναγνώριση ανωμαλιών στη χρονοσειρά αποκρίσεων του υπό μελέτη συστήματος. Για την αναγνώριση ανωμαλιών σε πραγματικά δεδομένα, σε πραγματικό χρόνο, υπάρχουν αλγόριθμοι όπως αυτοί που αναφέρονται στα \cite{repad}, \cite{lee2023repad2} και \cite{rerepad}. Τα τελευταία δύο βασίζονται στο RePAD \cite{repad} και αποτελούν αναβαθμισμένες εκδόσεις αυτού. Ο τρόπος λειτουργίας παραμένει κατά βάση ίδιος. Γίνεται εκπαίδευση ενός Long Short-Term Memory (LSTM) μοντέλου με τις προηγούμενες n μετρήσεις και κάθε φορά προβλέπεται η τιμή με βάση το προηγούμενο μοντέλο που υπολογίστηκε. Στη συνέχεια υπολογίζεται ένας δείκτης σχετικού λάθους και ένα όριο (threshold). Όταν το σχετικό λάθος ειναι μεγαλύτερο του threshold έχουμε εντοπισμό ανώμαλου σημείου. Στην απλή εκδοχή του RePAD για τον υπολογισμό του threshold χρησιμοποιούνται όλα τα προηγούμενα υπολογισμένα σχετικά λάθοι, κάτι το οποίο σε βάθος χρόνου θα καθιστά τον εντοπισμό αργό και σε ορισμένες περιπτώσεις μη λειτουργικό. Το πρόβλημα αυτό λύνεται στο RePAD2 \cite{lee2023repad2}, που το πλήθος των προηγούμενων σχετικών λαθών που συνυπολογίζονται είναι πεπερασμένο. Για να καλύψει τα κενά που αυτό δημιουργεί γίνονται περαιτέρω αλλαγές όπως ο επαναϋπολογισμός των μετρικών (σχετικό λάθος, threshold, προβλεπόμενη τιμή).

\begin{table}
	\begin{center}
		\caption{Σύγκριση Εργαλείων Ενεργής Παρακολούθησης}
		\label{tab:active_monitoring_characteristics}
		\begin{tabular}{| p{40mm} | c | c | c | c | c | c |}
			\hline & \thead{Τρέχουσα \\ Διπλωματική} & \thead{Better \\ Uptime} & \thead{Uptime \\ Robot} & \thead{Site24x7} & \thead{Uptimia} & \thead{Kuma} \tabularnewline
			\hline \thead{Σταθερότητα} & \checkmark & \checkmark & \checkmark & \checkmark & \checkmark & \checkmark \tabularnewline
			\hline \thead{Βάση Δεδομένων} & \checkmark  & \checkmark  & \checkmark & \checkmark & \checkmark & \tabularnewline
			\hline \thead{Ιστορικά Δεδομένα} & \checkmark & \checkmark & & & & \tabularnewline
			\hline \thead{Διαγράμματα} & \checkmark & \checkmark & & \checkmark & \checkmark & \tabularnewline
			\hline \thead{Ειδοποιήσεις} & \checkmark & \checkmark & \checkmark & \checkmark & \checkmark & \checkmark \tabularnewline
			\hline \thead{Ελαχιστοποίηση \\ Downtime} & \checkmark & \checkmark & \checkmark & \checkmark & \checkmark & \checkmark \tabularnewline
			\hline \thead{Παραμετροποίηση \\ μηνυμάτων} & \checkmark & \checkmark & \checkmark & \checkmark & \checkmark & \tabularnewline
			\hline \thead{Ρύθμιση Χρόνου \\ μεταξύ μηνυμάτων} & \checkmark & \checkmark & \checkmark & \checkmark & \checkmark & \checkmark \tabularnewline
			\hline \thead{Δυναμική \\ Ρύθμιση Πόρων} & \checkmark & & & & & \tabularnewline
            \hline \thead{Αναγνώριση \\ Ανωμαλιών} & \checkmark & & & & & \tabularnewline
            \hline
		\end{tabular}
	\end{center}
\end{table}
\newevenside

% % chapter 4 = Υλοποίηση του συστήματος
% 1η Υλοποίηση - Χρήση απλών schedulers και ενός express server που κρατούσαν όλα τα requests μόνο όσο "ζούσε" ο server.
% 2η Υλοποίηση - Προσθήκη στο προηγούμενο ενός κοινού σημείου (socket.io). Σε αυτή την υλοποίηση θα υπήρχαν 1server με τον οποιό θα "επικοινωνούσε" ο χρήστης και πολλοί server "clients" που θα μιλούσαν με τον αρχικό. Η χρήση websockets/socket.io θα ήταν εξαιρετική, καθώς θα μπορούσε σε κάποιες περιτπώσεις να οφελεί η αμφίφρομη εποικοινωνία server και worker, αλλά το γεγονός ότι οι περισσότερες επικοινωνίες θασ είναι μονόδρομες μάλλον μας ωθεί να χρησιμοποιήσουμε έναν κλασικό http server. Επίσης το scaling ενός websocket server είναι αρκετά πιο περίπλοκο και δύσκολο στη συντήρηση.
% 3η Υλοποίηση - Χρήση ενός ή περισοοτέρων (εφόσον χρειαστεί) server και ένας ή περισσότερη workers (scalable). Η κεντρικός server εξυπειρετεί μόνο την επικοινωνία μεταξύ worker και χρήστη και είναι υπεύθυνος για τη δημιουργία, καταστροφή, τροποποίηση ήδη υπάρχοντων jobs. Κάθε worker έχει το δικό του scheduler και εκτελεί τα δικά του jobs/apis καθώς και ένα cleanup jobs μία φορά τη βδομάδα που κάνει Upload τα raw responses, κάποια χρήσιμα στοιχεία σε ξεχωριστά αρχεία καθώς και υπολογίζει/επαναυπολογίζει χρήσιμες στατιστικές τιμές για κάθε μέρα (εδώ αναφορά στο πως περίπου λειτουργεί ο scheduler και εξήγηση της λειτουργίας του άμα "πέσει")

% 1ή υλοποίηση - η πιο απλή και μη λειτουργική. Ένας node server. Κάθε φορά που έρχεται qpi request σηκώνει καινούργιο setInterval με συγκεκριμένο interval κάθε φορά. Ιδανικό γιατί δεν θα έχεις ποτέ πρόβλημα με τον χρόνο
% 	μεταξύ διαδοχικών request επειδή είναι ακριβές. Κακή απόδοση όταν ο αριθμός των request αυξηθεί πάρα πολύ. Ξεχωριστά timer αυξάνουν την πολυπλοκότητα. Επίσης θα υπάρχει πρόβλημα όταν "πέφτει" ο server εξαιτίας κάποιου πιθανού
%	προβλήματος. Ακόμα και να κρατάμε τα jobs σε κάποια βάση δεν θα ξέρουμε πότε πρέπει να ξανατρέξουν. Αν πχ έχεις κάποιο
%	api να τρέχει κάθε 30s και μόλις εκτελεστεί "πεσει" ο server. Τότε θα πρέπει είτε να μην ξεκινήσεις το job και να χάσεις πιθανώς ένα request
% 	είτε να τρέξεις το job ότν και να γίνει και να έχεις διπλοεγγραφές.

% 2ή υλοποίηση - ένας node http και socketio/server και ένας ή παραπάνω socketio/client server. O main server θα είναι υπεύθυνος για το routing
%	των apis και οι υπόλοιποιν θα λαμβάνουν μηνυματα και θα τα εκτελούν ξεκινώντας πάλι setIntevral με συγκεκριμένα interval το καθένα
%	έτσι λύνται το πρόβλημα του πιθανού overload του main server. Ο λόγος χρήσης websockets είναι για να έχεις πιο γρήγορη και άμεση επικοινωνία
%	μεταξύ των server (worker και main) - επικοινωνία που εξυπηρετεί στην επισκόπηση του συστήματος (πόσα μηνύματα έχει κάθε worker) και μαζική μεταφορά μηνυμάτων
%	σε περίπτωση που πέσει κάποιος server. Επίσης websockets εξαιρετικά πιο γρήγορη σε σχέση με http.
%	επίσης επειδή το connection μεταξύ server και worker γίβεται μία φορά γλιτώνεις τα overhead από όλα τα πιθανά http requests που θα έκανες

% 3η υλοποίηση - ένας http server (ή παραπάνω αν χρειαστεί για καλύτερη εξυπηρέτηση) που απλώς servιρουν δεδομένα.
%	worker και server δεν επικοινωνούν μεταξύ τους, αλλά έχοιυν ένα κοινό΄σημείο αναφοράς (db).
% 	κάθε worker έχει ένα όνομα προκειμένου να ξέρει από που θα αντλήσει ποληροφορία
%	μετά ξεκίνα να μιλάς για το πως προκύπτουν τα στατιστικά. Μίλα για cleanup job και το πως ακριβώς λειτουργεί
% 	ανέφερε avro και το πως η standarized μορφή της πληροφορίας που αποθηκεύεται το καθιστά
%	μία πολύ καλή επιλογή για την αποθήκευση της πληροφορίας

\chapter{Υλοποιήσεις}
\label{chapter:implementations}

Στο κεφάλαιο αυτό θα αναφερθούμε στις τρεις διαφορετικές εκδόσεις του συστήματος που
υλοποιήσαμε μέχρι να καταλήξουμε στην τωρινή και τους λόγους που μας ωθούσαν να κάνουμε αυτές τις αλλαγές.

Πριν ξεκινήσουμε όμως θα πρέπει να ορίσουμε τις βασικές ενέργειες που θα εκτελεί ένα τέτοιο σύστημα.
Αρχικά θέλουμε να έχει δυνατότητες σέρβερ, ώστε να εξυπηρετεί απομακρυσμένους χρήστες που συνδέονται μέσω του δικτύου. 
Το σύστημα επιπροσθέτως θα πρέπει να μπορεί να στέλνει αιτήματα σε εξωτερικούς σέρβερ, ιστοσελίδες και γενικά 
εφαρμογές που επικοινωνούν με το διαδίκτυο. Σε αυτό το σημείο τονίζουμε την αναγκαιότητα ύπαρξης μίας βάσης δεδομένων που θα αποθηκεύει
πληρογορία σχετική με την ορθή λειτουργία του συστήματος. Τέλος και ίσως το πιο σημαντικό, απαιτείται η ύπαρξη κάποιας
μορφής scheduling, ενός μηχανισμού δηλαδή που θα ρυθμίζει πότε θα γίνονται τα αιτήματα και ποια από αυτά πρέπει γίνουν. 

Όλα αυτά αφορούν αποκλειστικά τη λειτουργία του backend του συστήματος. Πέρα από αυτά
θα πρέπει να υπάρχει μία γραφική διεπαφή μέσω της οποίας κάθε χρήστης θα μπορεί να δει τα δεδομένα που παράγονται
από τη χρήση του συστήματός.

Πλέον και χάριν συντομίας, στο υπόλοιπο κομμάτι της διπλωματικής αυτής εργασίας, θα αναφερόμαστε
στο σύστημα που δημιουργήσαμε, με το όνομα \textbf{Lychte} (/licht/). Η ονομασία αυτή προκύπτει από τα αρχικά του "\textbf{L}ightweight
\textbf{Y}et \textbf{C}onfigurable \textbf{H}ΤTP \textbf{T}raffic \textbf{E}xpert", που περιγράφει την λειτουργία και μόνο 
μερικά από τα πολλά χαρακτηριστικά σου συστήματος μας.  

\section{Version 0.0.1}
\label{section:first_implementation}

% 1ή υλοποίηση - η πιο απλή και μη λειτουργική. Ένας node σέρβερ. Κάθε φορά που έρχεται qpi request σηκώνει καινούργιο setInterval με συγκεκριμένο interval κάθε φορά. Ιδανικό γιατί δεν θα έχεις ποτέ πρόβλημα με τον χρόνο
% 	μεταξύ διαδοχικών request επειδή είναι ακριβές. Κακή απόδοση όταν ο αριθμός των request αυξηθεί πάρα πολύ. Ξεχωριστά timer αυξάνουν την πολυπλοκότητα. Επίσης θα υπάρχει πρόβλημα όταν "πέφτει" ο σέρβερ εξαιτίας κάποιου πιθανού
%	προβλήματος. Ακόμα και να κρατάμε τα jobs σε κάποια βάση δεν θα ξέρουμε πότε πρέπει να ξανατρέξουν. Αν πχ έχεις κάποιο
%	api να τρέχει κάθε 30s και μόλις εκτελεστεί "πεσει" ο σέρβερ. Τότε θα πρέπει είτε να μην ξεκινήσεις το job και να χάσεις πιθανώς ένα request
% 	είτε να τρέξεις το job ότν και να γίνει και να έχεις διπλοεγγραφές.

Η πρώτη και πιο απλή έκδοση του υπό κατασκευής συστήματος Ενεργής Παρακολούθησης. Αποτελείται από έναν
nodejs σέρβερ (και πιο συγκεκριμένα express σέρβερ), ο οποίος είναι υπεύθυνος για τον χειρισμό όλων των λειτουργιών της εφαρμογής. Αυτός συνδέεται
άμεσα με μία MongoDB που θα αποτελέσει τη βάση δεδομένων του συστήματος.

Ο σέρβερ αρχικά στεγάζει το api της εφαρμογής. Μέσα από αυτό κάθε χρήστης που έχει πρόσβαση στο σύστημα (σωστά credentials)
μπορεί να αντλεί πληροφορία ήδη αποθηκευμένη στη βάση ή να την τροποποιεί και να δημιουργεί νέα. Τα δεδομένα που μπορεί να παράξει
είναι περιορισμένα, βέβαια, καθώς υπάρχουν collections στα οποία γράφει μόνο το backend. Για να συνεχίσουμε και να δούμε το υπόλοιπο σύστημα
θα πρεπει να αναφερθούμε στο pipeline της εφαρμογής, στα βήματα δηλαδή που θα ακολουθήσει ο χρήστης εντός αυτής.

Κύρια λειτουργία του Lychte είναι η αυτόματη παρακολούθηση του χρόνου λειτουργίας και απόκρισης ενός ιστοτόπου.
Για να γίνει αυτό ο χρήστης θα πρέπει να εισάγει το url, που επιθυμεί να επιβλέπει και να ενημερώνεται σχετικά με
τη κατάστασή του, και ένα χρονικό διάστημα βάσει του οποίου θα γίνονται επαναλαμβανόμενα αιτήματα.
Έπειτα η πληροφορία αυτή μεταφέρεται στον σέρβερ υπό τη μορφή HTTP request. Αυτός το μήνυμα που περιέχει το url και
όποια ακόμα πληροφορία απαιτείται (headers, body) για να μπορέσει να κάνει την κλήση στο εξωτερικό σύστημα
που ο χρήστης ορίζει. Τέλος αποθηκεύει το αίτημα στη βάση δεδομένων και ξεκινάει έναν απλό scheduler
ο οποίος επαναλαμβάνει το αίτημα στο url που ορίστηκε από το χρήστη ανάλογα με τον διάστημα
που επιλέχθηκε.

Καταλήγουμε έτσι στη δεύτερη, μεγάλη, λειτουργία του σέρβερ, το scheduling. Κάθε φορά που αποθηκεύεται ένα καινούργιο
αίτημα για παρακολούθηση, ξεκινάει μία νέα διεργασία εντός αυτής που ήδη τρέχει (αυτής του σέρβερ). Η διεργασία παιδί (child process),
λαμβάνει κάποια πληροφορία σχετικά με το αίτημα που θα πρέπει να εκτελεί, από την διεργασία που την δημιούργησε. Η τελευταία μάλιστα
έχει τη δυνατότητα να σταματήσει τη λειτουργία όσων διεργασιών έχει ξεκινήσει αρκεί να γνωρίζει το pid που χαρακτηρίζει κάθε μία μονοσήμαντα.

\begin{figure}[!ht]
	\centering
	\includegraphics[width=0.8\textwidth]{./images/chapter4/lychte-first-implementation.png}
	\caption[Διάγραμμα πρώτης Υλοποίησης]{Διάγραμμα Πρώτης Υλοποίησης Συστήματος Ενεργής Παρακολούθησης. Ένας express σέρβερ, με μία mongo βάση δεδομένων εξυπηρετούν χρήστες, ενώ παράλληλα στέλνουν αιτήματα σε εξωτερικά συστήματα προς παρακολούθηση}
	\label{fig:first_implementation}
\end{figure}

Όλες οι διεργασίες εκτελούν την ίδια συνάρτηση τα ορίσματα της οποίας αποτελούν τα χαρακτηριστικά που ορίζει ο χρήστης
κατά τη διαδικασία δημιουργίας του αιτήματος προς παρακολούθηση. Η συνάρτηση αυτή είναι υπεύθυνη για να κάνει την κλήση στο εξωτερικό
σύστημα, να λάβει την απάντηση, να την αναλύσει (ανάλογα με το είδος της αναμενόμενης απάντησης) και να την αποθηκεύσει,
μαζί με κάποιες ακόμα χρήσιμες πληροφορίες, στη βάση δεδομένων. Πέρα από το response που μπορεί να είναι απλό κείμενο (html), αντικείμενο (json) ή κάποιο
αρχείο αποθηκεύονται και οι χρονισμοί του αιτήματος, το πότε δηλαδή ξεκίνησε το αίτημα, πότε τελείωσε καθώς και το χρόνο που μεσολάβησε μεταξύ τους. Το τελευταίο
θα μπορούσε να υπολογιστεί και εκ των υστέρων σε κάθε αίτημα που το χρειάζεται, καθώς είναι δεδομένο που προκύπτει από αυτά που ήδη υπάρχουν.
Εξαιτίας όμως της φύσης του συστήματος και των δεδομένων που μαζεύονται
θεωρούμε ότι κρίνεται αναγκαίο να κερδίσουμε επεξεργαστική ισχύ αποφεύγοντας υπολογισμούς 
για δεδομένα που γνωρίζουμε ότι θα ζητούνται συνεχώς, ώστε να εξηπυρετούνται όσο το δυνατό
πιο γρήγορα και άμεσα τα αιτήματα του χρήστη ως προς τον σέρβερ

Ένα τέτοιο σύστημα μπορεί να λειτουργήσει όπως ακριβώς περιγράψαμε παραπάνω. Έχει όμως κάποια βασικά μειονεκτήματα.
Αρχικά δεν υπάρχει κάποιο κοινό σημείο αναφοράς (single point of reference). Όσες διεργασίες υπάρχουν μέσα στο σέρβερ κοινούνται αυτόνομα,
χωρίς να υπάρχει κάποιος τρόπος να αναφερθείς σε κάποια αν δεν γνωρίζεις εκ των προτέρων το pid της και αν δεν έχεις πρόσβαση
στο σύστημα που τις εκτελεί. Επίσης το χαρακτηριστικό αυτό των διεργασιών έχει υπόσταση μόνο εντός του ιδίου περιβάλλοντος εκτέλεσης. Ακόμα και
να αποθηκεύαμε το pid της διεργασίας που είναι υπεύθυνο για την εκτέλεση ενός συγκεκριμένου αιτήματος, δεν θα μπορούσαμε να το χρησιμοποιήσουμε 
κάπως σε περίπτωση που επανεκκινηθεί το υπολογιστικό σύστημα που φιλοξενεί τον σέρβερ, καθώς οι καινούργιες διεργασίες θα είχαν διαφορετικό id.
Οι "καινούργιες διεργασίες" εδώ χαρακτηρίζουν ένα σύνολο διεργασιών που θα ξεκινούσε ο σέρβερ μέχρι όλα τα αιτήματα που βρίσκονται αποθηκευμένα στη βάση
να ξεκινήσουν να ελέγχονται από κάποιο process.

Αξίζει ακόμα να σημειωθεί ότι το σύστημα αυτό δεν μπορεί να κλιμακωθεί εύκολα, και αποδοτικά, καθώς 
για να επεκτείνεις το σύνολο των αιτημάτων που μπορείς να διαχειριστείς θα πρέπει να συντηρείς έναν δεύτερο σέρβερ που ίσως να μην χρειάζεται,
ώστε να χωρέσουν παραπάνω διεργασίες. Το πρόβλημα αυτό όμως μπορεί να λυθεί σπάζοντας τις δύο βασικές λειτουργίες του
Lychte σε δύο διακριτές και σαφώς ανεξάρτητες οντότητες. Ενός scheduler και ενός σέρβερ. Αυτό ακριβώς θα δούμε καινούργιες
στη δεύτερη υλοποίηση του συστήματός μας.

\newpage

\section{Version 0.1.0}
\label{section:second_implementation}

Συνεχίζοντας τη συλλογιστική πορεία της πρώτης υλοποίησης, καλούμαστε να λύσουμε ένα από τα πιο βασικά προβλήματα, που προέκυψαν
Ο διαμοιρασμός των λειτουργιών. Όπως προαναφέραμε θα πρέπει να υπάρχει ένας (ή και παραπάνω) σέρβερ που θα εξυπηρετεί
τους χρήστες μέσω διεπαφών των προγραμματιστικών διαδικασιών που παρέχει, και ένας (ή παραπάνω) scheduler, που θα εκτελούν
αποκλειστικά και μόνο αιτήματα προς τα συστήματα που θέλουμε να ελέγχουμε. Ας δούμε όμως πιο συγκεκριμένα το ανανεωμένο σύστημα (\autoref{fig:second_implementaion})

\begin{figure}[!ht]
	\centering
	\includegraphics[width=0.8\textwidth]{./images/chapter4/lychte-second-implementation.png}
	\caption[Διάγραμμα δεύτερης Υλοποίησης]{Διάγραμμα Δεύτερης Υλοποίησης Συστήματος Ενεργής Παρακολούθησης. Αποτελείται στο κέντρο του από έναν express σέρβερ και μία mongo βάση δεδομένων. Ο σέρβερ επικοινωνεί με δύο schedulers οι οποίοι είναι υπεύθυνοι για τον έλεγχο εξωτερικών προς το σύστημα εφαρμογών/σέρβερ}
	\label{fig:second_implementaion}
\end{figure}

Το βασικό συστατικό του Lychte, ο σέρβερ δηλαδή παραμένει όπως είχε. Έχει δηλαδή routes μέσα από τα οποία εξυπηρετεί αιτήματα
των χρηστών που προέρχονται από το frontend του συστήματος. Αυτά αφορούν την αποστολή πληροφορίας προς τους χρήστες προκειμένου να δουν
τα δεδομένα που αποθηκεύουμε στη βάση μας, αλλά και την αποθήκευση πληροφορίας που προέρχεται από τους χρήστες που σχετίζεται με τη δημιουργία
καινούργιων ελέγχων σε urls που επιθυμούν να παρακολουθούν.

Αυτό που αλλάζει σε σχέση με την προηγούμενη υλοποίηση, είναι ο τρόπος που διευθετεί
την εκτέλεση των επαναλαμβανόμενων αιτημάτων που πρέπει να κάνει προς τα εξωτερικά υπό παρακολούθηση συστήματα.
Όπως είπαμε νωρίτερα δεν είναι πλέον υπεύθυνος για το scheduling των ενεργειών που πρέπει να γίνουν, 
αλλά αντιθέτως στέλνει σε ένα άλλο σύστημα (scheduler) τις πληροφορίες που χρειάζεται προκειμένου να ρυθμιστεί και να οργανωθεί κατάλληλα
 ο έλεγχος των urls που θέλουμε να ελέγχονται.

Πλέον θα αναφερόμαστε στο σύστημα του scheduler ως worker, καθώς αυτό που κάνει είναι να λαμβάνει δεδομένα και να
εκτελεί βάση αυτών συνεχώς αιτήματα προς άλλα εξωτερικά συστήματα. Ο τρόπος λειτουργίας του είναι αρκετά παρόμοιος
με τη αυτή του scheduler της πρώτης υλοποίησης, με κάποιες μικρές διαφορές. Κάθε φορά που λαμβάνει κάποιο αίτημα από τον κεντρικό σέρβερ, κάνει ένα δοκιμαστικό request
στο url που πρόκειται να ελεγχθεί, για να διαπιστωθεί η ορθότητά του. Έπειτα ξεκινάει έναν timer που εκτελεί το αίτημα ανά
χρονικό διάστημα που ορίζει ο χρήστης. Ο χρόνος αυτός μπορεί να φτάνει ένα κάτω όριο του ενός δευτερολέπτου, αλλά για λόγους σταθερότητας και
εγγύησης ότι η εκτέλεση κάποιου αιτήματος δεν θα μπλοκάρει την εκτέλεση κάποιου άλλου, έχουμε βάλει ένα κάτω όριο των 30 δευτερολέπτων στην υλοποίησή μας.
Στην περίπτωση που το αίτημα χρειαστεί χρόνο, για να ικανοποιηθεί, μεγαλύτερο αυτού που έχει οριστεί σαν χρόνος μεταξύ
διαδοχικών αιτημάτων διακόπτουμε το αίτημα μέσω timeouts και το αποθηκεύουμε στη βάση σαν αποτυχημένη απόκριση (failed response).
Αυτό γίνεται για να εξασφαλίσουμε ότι το χρονικό διάστημα μεταξύ διαδοχικών αιτημάτων θα παραμένει
σχετικά σταθερό και η απόλυτη μετατόπιση του χρόνου που υπάρχει μεταξύ των αιτημάτων σε βάθος χρόνου. 
δεν θα αλλάζει σε μεγάλο βαθμό.

Για να γίνει πιο κατανοητό το παραπάνω αρκεί να μελετήσουμε τα \autoref{fig:perfect_request_cycle} και \autoref{fig:timeouts_request_cycle}

Το πρώτο περιγράφει την εκτέλεση ενός αιτήματος ανά ένα χρονικό διάστημα ενός λεπτού. Όπως παρατηρούμε το
αίτημα εκτελείται σε αυτό το μήκος χρόνου με αποτέλεσμα, όταν ξαναέρθει η στιγμή που πρέπει να επαναεκτελεστεί,
αυτό να γίνει χωρίς περιττές καθυστερήσεις. Αν υποθέσουμε ότι ο χρόνος που μεσολάβησε μεταξύ του πρώτου και του δεύτερου αιτήματος
είναι ακριβώς εξήντα δευτερόλεπτα μπορούμε να πούμε ότι η απόκλιση, σε σχέση με τον ένα λεπτό, μεταξύ των αυτών είναι μηδενική.
Βέβαια κάτι τέτοιο δεν υφίσταται στην πραγματικότητα καθώς οι timers έχουν πάντα μία μικρή απόκλιση, προς τα πάνω ή προς τα κάτω, της τάξης των nanoseconds.
Σε βάθος χρόνου η απόκλιση παραμένει κοντά στο μηδέν.

\begin{figure}[!ht]
	\centering
	\includegraphics[width=0.8\textwidth]{./images/chapter4/perfect_request_cycle.png}
	\caption[Κύκλος Ζωής εκτέλεσης ενός επαναλαμβανόμενου Αιτήματος]{Κύκλος Ζωής εκτέλεσης ενός επαναλαμβανόμενου Αιτήματος}
	\label{fig:perfect_request_cycle}
\end{figure}

Έπειτα ας μελετήσουμε την περίπτωση που το αίτημα χρειάζεται παραπάνω χρόνο για να εκτελεστεί (εξαιτίας καθυστερήσεων του εξωτερικού συστήματος που ελέγχεται).
Στο \autoref{fig:timeouts_request_cycle} μπορούμε να δούμε ακριβώς πως θα λειτουργούσε ένας worker στην περίπτωση που υπήρχαν μεγάλες εξωγενείς καθυστερήσεις, χωρίς τη χρήση
timeous αλλά και με τη χρήση αυτών. Όταν αναφερόμαστε σε timeouts περιγράφουμε την πρόωρη έξοδο από τη συνάρτηση που τρέχουμε, στην
προκειμένη, η εκτέλεση του αιτήματος. Παρατηρούμε ότι στα δύο χρονικά διαγράμματα έχουμε αρκετά μεγάλες διαφορές.
Στο πρώτο περιμένουμε να εκτελεστεί πλήρως το αίτημα, ακόμα και αν αυτό αργήσει παραπάνω από όσο
θα έπρεπε. Το επόμενο αίτημα ξεκινάει με κάποια καθυστέρηση ως προς το πρώτο και παρομοίος τα επόμενα στη σειρά.
Αυτή η διαδικασία εισάγει απόκλιση στους χρόνους των κλήσεων μεταξύ των αιτημάτων που υπό φυσιολογικές συνθήκες αναμένουμε να ισαπέχουν (χρονικά) μεταξύ τους.
Απόκλιση η οποία μάλιστα μπορεί συνεχώς να αυξάνεται, καθιστώντας το σύστημά μας αναξιόπιστο ως προς
το χρόνο εκτέλεσης και την ακρίβεια των αιτημάτων. Για αυτό το λόγο κρίνουμε αναγκαία τη χρήση timeouts.
Το μεινέκτημα σε αυτή την περίπτωση είναι ότι αποθηκεύουμε ως εσφαλμένο ένα πιθανώς σωστό response, λόγω του χρόνου που πήρε για να εκτελεστεί.

\begin{figure}[!ht]
	\centering
	\includegraphics[width=0.8\textwidth]{./images/chapter4/timeout_request_cycle.png}
	\caption[Σύγκριση χρήσης ή μη timeouts στον κύκλο ζωής εκτέλεσης ενός αργού επαναλαμβανόμενου αιτήματος]{Σύγκριση χρήσης ή μη timeouts στον κύκλο ζωής εκτέλεσης ενός αργού επαναλαμβανόμενου αιτήματος}
	\label{fig:timeouts_request_cycle}
\end{figure}

Μία ακόμα αλλαγή ως προς την πρώτη υλοποίηση αφορά την επικοινωνία μεταξύ σέρβερ και worker. Προκειμένου το σύστημα να
μπορεί να σπάσει σε πιο διακριτές έννοιες χάνουμε το πλεονέκτημα του να έχουμε όλες τις λειτουργίες εντός του ίδιου συστήματος.
Λειτουργίες οι οποίες πλέον για να επιτελούνται πρέπει να υπάρχει ένα τρόπος επικοινωνίας μεταξύ των εμπλεκόμενων.
Η επικοινωνία αυτή δεν θα είναι απαραίτητα μονόπλευρη. Η βασική λειτουργία όπως είχαμε δει και από την προηγούμενη υλοποίηση
είναι η αποστολή μηνυμάτων στον scheduler, σε αυτή την περίπτωση στον worker. Επεκτείνοντας όμως τις δυνατότητες του συστήματος
θα μπορούσαμε να έχουμε αμφίδρομη επικοινωνία μεταξύ σέρβερ και worker, ώστε να μπορεί να ενημερώνεται ο σέρβερ (όταν το ζητήσει) για την
κατάσταση των διαφόρων worker που είναι συνδεμένοι σε αυτόν. Κάτι τέτοιο θα είχε νόημα αν είχαμε παραπάνω από έναν worker και θα θέλαμε να κάνουμε
διαμοιρασμό φόρτου σε καθέναν από αυτούς. Αρχικά θα ρωτούσε όλους τους συνδεδεμένους worker σχετικά με το πλήθος των αιτημάτων
που ελέγχουν τη δεδομένη χρονική στιγμή. Έπειτα κάθε κόμβος θα απαντούσε και αυτός με το λιγότερο φόρτο θα επιλεγόταν ως ο κατάλληλος 
για τη δημιουργία μίας ακόμα διεργασίας ελέγχου ενός url.

Λόγω της αμφίδρομης σχέσης που έχουν τα δύο συστήματα μεταξύ τους, επιλέχθηκε το πρωτόκολλο επικοινωνίας των WebSockets.
Πιο συγκεκριμένα ο τρόπος ενσωμάτωσης της τεχνολογίας αυτής πραγματοποιήθηκε με τη χρήση της βιβλιοθήκης socket.io που αποτελεί ένα
abstract layer πάνω από το πρωτόκολλο WebSockets. Για να υπάρξει επικοινωνία θα πρέπει να υπάρχει από τη μία πλευρά ένας \textbf{socket.io-server} και από την άλλη, ένας \textbf{socket.io-client}.
Στη δική μας εφαρμογή ο socket.io-server θα είναι ο σέρβερ και ο socket.io-client ο worker. Κάθε φορά που θα ξεκινάει τη λειτουργία του ένας σέρβερ θα δημιουργεί στο τοπικό δίκτυο του
έναν socket server στον οποίο μπορούν να συνδεθούν και να επικοινωνήσουν όσοι διαθέτουν το url και τα απαραίτητα credentials. Από την άλλη
κάθε φορά που ξεκινάει τη λειτουργία του ένας worker θα πρέπει να μπορεί να συνδέεται κατευθείαν στο socket server που υπάρχει.
Θα πρέπει δηλαδή να διαθέτει από πριν το url που οδηγεί στον socket server του σέρβερ.

Οι λόγοι που επιλέχθηκαν τα WebSockets αντί ενός κλασσικού HTTP σέρβερ είναι κυρίως η ταχύτητα που προσφέρουν. Υπό περιτπώσεις
μπορεί να είναι μέχρι και δέκα φορές πιο γρήγορα σε σχέση με το συμβατικό πρωτόκολλο HTTP. Επίσης από τη στιγμή που κάθε worker 
θα επικοινωνεί αποκλειστικά με έναν (ή περισσότερους) βλέπουμε ότι δεν έχει νόημα η δημιουργία ολόκληρου σέρβερ στο πλαίσιο ενός worker.

\begin{figure}[!ht]
	\centering
	\includegraphics[width=0.8\textwidth]{./images/chapter4/socket.io-communication.png}
	\caption[Επικοινωνία σέρβερ-worker μέσω websockets]{Επικοινωνία σέρβερ-worker μέσω websockets}
	\label{fig:socketio-communitation}
\end{figure}

Η παραπάνω υλοποίηση έχει πολλά πλεονεκτήματα σε σχέση με την πρώτη. Αρχικά με τον καταμερισμό
των δύο βασικών λειτουργιών κερδίζουμε όσων αφορά την πολυπλοκότητα του συστήματος, κάτι το οποίο θα πρέπει να μας απασχολεί,
καθώς όσο πιο πολύπλοκο είναι ένα σύστημα τόσο πιο δύσκολα συντηρείται στη συνέχεια. Πέρα από αυτό, που αφορά
κυρίως τη διαδικασία ανάπτυξης λογισμικού και δεν επιφέρει κάποιο άμεσο κέρδος στους χρήστες, θα πρέπει να αναφέρουμε
ότι το σύστημα γενικά είναι πιο αποδοτικό, καθώς οι διεργασίες που κάθε worker στεγάζει είναι λιγότερο πιθανό να
κολλήσουν και να μπλοκάρουν λόγω φόρτου στο μηχάνημα που εκτελούνται. Τα requests γενικά δεν καταναλώνουν πολλούς πόρους από το σύστημα στο οποίο εκτελούνται,
σε αντίθεση με έναν server που μπορεί να δέχεται συνεχώς αιτήματα. Αξίζει να σημειωθεί ότι το api που παρέχει ο server είναι υπεύθυνο
για την αποστολή μεγάλου πλήθος πληροφορίας συνεχώς, που χρησιμοποιείται σε πληθώρα διαγραμμάτων, καθιστώντας τον έτσι απαιτητικό ως προς τους πόρους του
συστήματος. 

Ένας ακόμα τομέας στον οποίο συνεισφέρει σε μεγάλο βαθμό η διάσπαση των λειτουργιών αφορά την κλιμάκωση, και βασικά την οριζόντιο κλιμάκωση
του συστήματος. Αν δούμε ότι ο σέρβερ υστερεί και δυσκολεύεται να ανταπεξέλθει στα αιτήματα των χρηστών,
μπορούμε να φτιάξουμε έναν καινούργιο σέρβερ. Από την άλλη, αν παρατηρήσουμε ότι οι worker αργούν ή έχουν υπερφορτωθεί
μπορούμε να δημιουργήσουμε workers.

Ακόμα και με αυτή την υλοποίηση υπάρχει έλλειψη ενός κοινού σημείου αναφοράς. Ενός κόμβου στο σύστημά μας του οποίου
οι αλλαγές θα επηρέαζαν άμεσα τη λειτουργία όλων των οντοτήτων αυτού.  
\newevenside

% % chapter 5 = Επίδειξη Συστήματος
% - Παρουσίαση εικόνων από website
% - Εξήγεισαι τι δείχνουν και λειτουργικότητα

% ****ακράιες συνθήκες λειτουργίας συστήματος - πειράματα*******
% 	σε local βάση (πειράματα για 1ώρα) - κράτα σε αρχεία usage statiustics για το node process (ram, cpu, ....) ανά λεπτό
%	1) 50 apis σε ένα worker
%	2) 100 apis σε ένα worker
%	3) 200 apis σε ένα worker
%	4) 400 apis σε ένα worker
%	5) 400 apis σε ένα worker
% για καθένα από τα παραπάνω υπολόγιζε μέσο βαθμό απόκλισης από το expected interval και βάλε από δίπλα αποτελέσματα ram usage και cpu usage των threads εφόσον γίνεται


\newevenside
%\newevenside

\renewcommand{\bibname}{Βιβλιογραφία}
\bibliography{
	./bibliography/chapter1.bib,
	./bibliography/chapter2.bib,
	./bibliography/chapter3.bib
}

\end{document}

