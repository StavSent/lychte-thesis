{\fontfamily{cmr}\selectfont

\phantomsection
\addcontentsline{toc}{section}{Abstract}


\begin{center}
  \centering
  \textbf{\Large{Title}}
  \vspace{0.5cm}

  \textbf{\large{Development of an Active Monitoring System for Web Applications}}
  \vspace{1cm}

  \centering
  \textbf{Abstract}
\end{center}

The continuous digitization of daily life, has led to a plethora of internet applications developed within this framework. These data make it imperative, to have systems that will control the smooth operation of such software applications, which function as services (SaaS - Software as a Service).

Control checks can be exerted at various levels, from unit tests within the software development cycle (continuous integration, continuous deployment cycle) to Network Monitoring during the actual system operation.

While there are several tools for application control during the software development phase, there are not similarly many user-friendly and capable tools for control during the operational phase. Therefore, this thesis focuses on developing an Intelligent Monitoring system and, by extension, an application that will allow users to easily monitor the smooth operation of their internet pages, whether they are applications or simple static pages, to recognize and detect abnormal operation and to automatically modify the available computational resources in real-time.

The system relies on the basic method of detecting the availability of a website, known as ping. By pinging, we obtain useful information about whether the system under study can respond correctly to the requests it receives and the time it takes to respond. Continuing the logical progression of such system, we can even modify the message we send with predetermined data, to check the response of the system and verify response, thus having an additional mechanism of identifying and avoiding potential bugs or errors during the software’s development process.

Furthermore, by analyzing the sequence of timestamps generated by continuous requests to the system under study, we can extract significant information about its evolution over time, and this way identify abnormal behaviours. Finally, if the system we study is built as a Docker container, we can even dynamically modify the available resources, in order to operate optimally even under high load conditions.


\begin{flushright}
  \vspace{2cm}
  Sentonas Stavros
  \\
  Electrical \& Computer Engineering Department,
  \\
  Aristotle University of Thessaloniki, Greece
  \\
  June 2023
\end{flushright}

}
