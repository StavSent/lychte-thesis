{\fontfamily{cmr}\selectfont

\phantomsection
\addcontentsline{toc}{section}{Abstract}


\begin{center}
  \centering
  \textbf{\Large{Title}}
  \vspace{0.5cm}

  \textbf{\large{Development of a System for Uptime Status Monitoring}}
  \vspace{1cm}

  \centering
  \textbf{Abstract}
\end{center}

The evolution of technology and the abundance of applications developed within this framework make it imperative to have systems that will control their smooth operation. More specifically, we are talking about monitoring in the realm of the internet and the structures implemented on it.

Today we refer to a constantly growing and extensive network of web applications - software as a service (SaaS) - that reside on the Internet (World Wide Web), whose operations can be monitored in various ways. From unit testing, within the software development cycle (continuous integration, continuous deployment) to ensure that the system functions are running properly and avoid bugs, to network monitoring, to verify the correct functioning of the systems throughout their life cycle.

This thesis focuses on the development of a Network Monitoring system and a web application that will enable its users to easily monitor how their systems operate, whether they are applications or simply static pages. The system is based on the basic method of checking the availability of a website, known as ping. By pinging, we can obtain useful information regarding whether the system under study can respond and, if so, the time it takes to respond. Continuing the logical progression of such a system, we can further enhance the message we send with predetermined data to check the response of the system and verify the returned data, thereby identifying and avoiding potential bugs or errors during the software's development process

\begin{flushright}
  \vspace{2cm}
  Sentonas Stavros
  \\
  Electrical \& Computer Engineering Department,
  \\
  Aristotle University of Thessaloniki, Greece
  \\
  June 2023
\end{flushright}

}
