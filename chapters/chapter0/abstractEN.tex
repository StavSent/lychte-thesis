{\fontfamily{cmr}\selectfont

\phantomsection
\addcontentsline{toc}{section}{Abstract}


\begin{center}
  \centering
  \textbf{\Large{Title}}
  \vspace{0.5cm}

  \textbf{\large{Development of a System for Uptime Status Monitoring}}
  \vspace{1cm}

  \centering
  \textbf{Abstract}
\end{center}

The continuous digitization of everyday life, has led to a plethora of internet applica\hyp{}tions
that are developed within this framework. These data make it imperative, to have systems
that will check the smooth operation of such software applications, which function
as services (SaaS - Software as a Service)

Controle checks can be done at various levels. From Unit Tests, within the software's development
cycle (continuous integration, continuous deployment cycle), to Network Monitoring in the actual
system's operation.  

While there are several tools for application control checks, during the software development
phase, there are no equivalent user-friendly tools for control checks during the software operation
phase. Therefore, this thesis focuses on the development of an Active Network monitoring system and, by extension,
an application, that will enable users to easily monitor the smooth operation of
their web pages, whether they are applications or simply static web pages. 

The system is based on the basic method of detecting the availability of a website,
known as ping. By pinging, we can obtain useful information about, whether the system under study,
can properly respond to incoming requests and the time it takes  to respond. Continuing 
the logical flow of such a sustem, we can further enhance the sent message, with predetermined
data, to check the response of the system and verify the returned data, thus having an additional mechanism
of identifying and avoiding potential bugs or errors during the software's development process.

\begin{flushright}
  \vspace{2cm}
  Sentonas Stavros
  \\
  Electrical \& Computer Engineering Department,
  \\
  Aristotle University of Thessaloniki, Greece
  \\
  June 2023
\end{flushright}

}
