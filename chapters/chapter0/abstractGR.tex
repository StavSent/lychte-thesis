\begin{center}
  \centering

  \vspace{0.5cm}
  \centering
  \textbf{\Large{Περίληψη}}
  \phantomsection
  \addcontentsline{toc}{section}{Περίληψη}

  \vspace{1cm}

\end{center}

  Η συνεχόμενη ψηφιοποίηση της καθημερινότητας έχει οδηγήσει σε μία πληθώρα διαδικτυακών εφαρμογών
  που αναπτύσσονται στο πλαίσιο αυτής. Τα δεδομένα αυτά, καθιστούν επιτακτική την ανάγκη ύπαρξης συστημάτων
  που θα ελέγχουν την εύρυθμη λειτουργία τέτοιων εφαρμογών λογισμικού, που λειτουργούν ως
  υπηρεσίες (SaaS - Software as a Service).
  
  Ο έλεγχος μπορεί να γίνει σε διάφορα επίπεδα. Από ελέγχους μονάδων (Unit Tests), στο πλαίσιο
  του κύκλου ανάπτυξης του λογισμικού (continuous integration, continuous deployment cycle), μέχρι την
  Παρακολούθηση Δικτύου (Network Monito\hyp{}ring), στην πραγματική λειτουργία του συστήματος.  
  
  Ενώ υπάρχουν αρκετά εργαλεία για τον έλεγχο εφαρμογών, κατά τη φάση ανάπτυξης του λογισμικού, δεν υπάρχουν αντίστοιχα εύχρηστα και ικανά εργαλεία για έλεγχο στη φάση λειτουργίας αυτού. Η παρούσα, λοιπόν, διπλωματική εστιάζει στην ανάπτυξη ενός συστήματος Έξυπνης Παρακολούθησης και κατ' επέκταση εφαρμογής που θα παρέχει τη δυνατότητα στους χρήστες της να παρακολουθούν εύκολα την ομαλή λειτουργία των διαδικτυακών σελιδών τους, είτε αυτά είναι εφαρμογές, είτε απλά στατικές σελίδες, να κάνουν αναγνώριση και εντοπισμό μη φυσιολογικής λειτουργίας, καθώς και να τροποποιούν αυτόματα, τους διαθέσιμους στα υπολογιστικά συστήματα, πόρους σε πραγματικό χρόνο.
  
  Το σύστημα στηρίζεται στη βασική μέθοδο εντοπισμού διαθεσιμότητας μίας ιστοσελίδας, γνωστή και
  ως ping. Κάνοντας ping μπορούμε να πάρουμε χρήσιμη πληροφορία σχετικά με το αν το υπό μελέτη σύστημα
  μπορεί να αποκριθεί ορθά στα αιτήματα που δέχεται, και σχετικά με το χρόνο που χρειάστηκε προκειμένου
  να απαντήσει. Συνεχίζοντας την λογική πορεία ενός τέτοιου συστήματος μπορούμε ακόμα στο μήνυμα που στέλνουμε να έχουμε πληροφορία που θα επηρεάζει την απάντηση που θα περιμέναμε να δούμε, έχοντας έτσι έναν ακόμα μηχανισμό για την αναγνώριση και αποφυγή πιθανών bugs, ή λαθών κατά τη διαδικασία ανάπτυξης λογισμικού ως υπηρεσία.
  
  Επιπλέον αναλύοντας την ακολουθία των χρονισμών, που παράγεται από τις συνεχείς κλήσεις στο υπό μελέτη σύστημα, μπορούμε να αντλήσουμε σημαντικά στοιχεία για την εξέλιξη της στο χρόνο και έτσι να εντοπίσουμε μη φυσιολογικά σημεία πάνω σε αυτή. Τέλος αν το σύστημα που μελετάμε αποτελεί docker container, προσφέρεται η δυνατότητα δυναμικής τροποποίησεις των διαθέσιμων πόρων, προκειμένου να ανταποκρίνεται και να λειτουργεί με βέλτιστο τρόπο ακόμα και υπό συνθήκες υψηλού φορτίου.
