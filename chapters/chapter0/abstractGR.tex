\begin{center}
  \centering

  \vspace{0.5cm}
  \centering
  \textbf{\Large{Περίληψη}}
  \phantomsection
  \addcontentsline{toc}{section}{Περίληψη}

  \vspace{1cm}

\end{center}

  Η εξέλιξη της τεχνολογίας και της πληθώρας εφαρμογών που αναπτύσσονται στα πλαίσιο αυτής, καθιστούν επιτακτική την ανάγκη ύπαρξης συστημάτων που θα ελέγχουν την εύρυθμη λειτουργία τους. Πιο συγκεκριμένα μιλάμε για την ελέξιλη στο χώρο του διαδικτύου και των δομών που έχουν υλοποιηθεί πάνω σε αυτό.
  
  Πλέον αναφερόμαστε σε ένα συνεχώς αυξανόμενο και ευρύ δίκτυο web εφαρμογών - λογισμικών ως υπηρεσίας (SaaS - Software as a Service) που ζουν στον Διαδίκτυο (World Wide Web). H λειτουργία αυτών μπορεί να ελεχθεί με διάφορους τρόπους. Από Unit Testing, στο πλαίσιο του κύκλου ανάντυξης του λογισμικού (continuous integration, continuous deployment cycle) προκειμένου να ελεχθεί λειτουργικά το σύστημα για την αποφυγή bugs, μέχρι και Παρακολούθηση Δικτύου (Network Monitoring), για να επιβεβαιωθεί η σωστή λειτουργία των συστημάτων καθόλη της διάρκεια του κύκλου ζωής τους.
  
  Η παρούσα διπλωματική εστιάζει στην ανάπτυξη ενός συστήματος Παρακολούθησης Δικτύου και κατεπέκταση εφαρμογής που θα δίνει της δυνατότητα στους χρήστες της να παρακολουθούν, εύκολα, την ομαλή λειτουργία των διαδικτυακών σελιδών τους, είτε αυτά είναι εφαρμογές, είτε απλά στατικές σελίδες. Το σύστημα στηρίζεται στη βασική μέθοδο εντοπισμού διαθεσιμότητας μίας ιστοσελίδας, γνωστή και ως ping. Κάνοντας ping μπορούμε να πάρουμε χρήσιμη πληροφορία σχετικά με το αν το υπό μελέτη σύστημα μπορεί να ανταποκριθεί και εφόσον ανταποκριθεί σχετικά με το χρόνο που μεσολάβησε μέχρι να απαντήσει. Συνεχίζοντας την λογική πορεία ενός τέτοιου συστήματος μπορούμε ακόμα στο μήνυμα που στέλνουμε να έχουμε πληροφορία που θα επηρεάζει την απάντηση που θα περιμέναμε να δούμε, έχοντας έτσι έναν ακόμα μηχανισμό για την αναγνώριση και αποφυγή πιθανών bugs, ή λαθών κατά τη διαδικασία ανάππτυξης λογισμικών ως υπηρεσία.
