\break

\section{Στατιστικές μετρικές}
\label{section:statistics}

Στην υποενότητα αυτή θα δούμε τον τρόπο υπολογισμού μερικών κλασικών μετρικών της στατιστικής
που χρησιμοποιούνται για την εξαγωγή συμπερασμάτων στο σύνολο των δεδομένων που αποθηκεύουμε.
Σε όλους τους παρακάτω τύπους έχουμε δεδομένα $x_i$ όπου, $i = 0, ..., k$

\begin{enumerate}
	\item \textbf{Αριθμητική Μέση Τιμή}:
	      \begin{equation}
		      mean(x) = \frac{\sum_{n = 1}^{k} x_n}{k}
	      \end{equation}
	\item \textbf{Διάμεσος}:
	      \begin{equation}
		      median(x) =
		      \begin{cases}
			      x_{\frac{n + 1}{2}}                             & \text{για n περιττό}
			      \\[10pt]
			      \frac{x_{\frac{n}{2}} + x_{\frac{n}{2} + 1}}{2} & \text{για n ζυγό}
		      \end{cases}
	      \end{equation}
	\item \textbf{Τυπική Απόκλιση Πληθυσμού}:
	      \begin{equation}
		      std(x) = \sqrt{\frac{\sum_{i = 1}^{n} (x_i - mean(x))^2}{n - 1}}
	      \end{equation}
	\item \textbf{Τεταρτημόρια}:
	      \begin{equation}
				\begin{cases}
					q1(x) = mean(x_i) & \text{όπου i = 0, ...., k/2} \\
					q3(x) = mean(x_j) & \text{όπου j = k/2 + 1, ...., k}
				\end{cases}
	      \end{equation}
\end{enumerate}

\break