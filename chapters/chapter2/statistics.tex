\break
\section{Στατιστικές μετρικές}
\label{section:statistics}

Στην υποενότητα αυτή θα δούμε τον τρόπο υπολογισμού μερικών κλασικών μετρικών της στατιστικής
που χρησιμοποιούνται για την εξαγωγή συμπερασμάτων στο σύνολο των δεδομένων που αποθηκεύουμε.
Σε όλους τους παρακάτω τύπους έχουμε δεδομένα $x_i$ όπου, $i = 0, ..., k$

\begin{enumerate}
	\item \textbf{Αριθμητική Μέση Τιμή}: Αποτελεί μία από τις πιο βασικές μετρικές που χρησιμοποιείται ευρέως σε στατιστικές μελέτες.
		Περιγράφει την τάση του εκάστοτε συνόλου δεδομένων που μελετάμε να κυμαίνεται γύρω από μία τιμή.
	      \begin{equation}
		      mean(x) = \frac{\sum_{n = 1}^{k} x_n}{k}
	      \end{equation}
	\item \textbf{Διάμεσος}: Είναι η τιμή που διαχωρίζει το υψηλότερο μισό από το κάτω μισό ενός διατεταγμένου συνόλου δεδομένων 
	      \begin{equation}
		      median(x) =
		      \begin{cases}
			      x_{\frac{n + 1}{2}}                             & \text{για n περιττό}
			      \\[10pt]
			      \frac{x_{\frac{n}{2}} + x_{\frac{n}{2} + 1}}{2} & \text{για n ζυγό}
		      \end{cases}
	      \end{equation}
	\item \textbf{Τυπική Απόκλιση Πληθυσμού}: Περιγράφει το κατά πόσο απέχουν το σύνολο των δεδομένων, κατά μέση τιμή, από τον \textbf{Μέσο Όρο}
	      \begin{equation}
		      std(x) = \sqrt{\frac{\sum_{i = 1}^{n} (x_i - mean(x))^2}{n - 1}}
	      \end{equation}
	\item \textbf{Τεταρτημόρια}: Όπως και η διάμεσος αποτελούν τις τιμές που διαχωρίζουν το σύνολο, των
		  διατεταγμένων κατά αύξουσα σειρά, δεδομένων στο 25\% και 75\% αντίστοιχα  
	      \begin{equation}
				\begin{cases}
					q1(x) = mean(x_i) & \text{όπου i = 0, ...., k/2} \\
					q3(x) = mean(x_j) & \text{όπου j = k/2 + 1, ...., k}
				\end{cases}
	      \end{equation}
\end{enumerate}

\break