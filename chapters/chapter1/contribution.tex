\section{Σκοπός - Συνεισφορά της Διπλωματικής Εργασίας}
\label{section:contribution}

% Η παρούσα διπλωματική εργασία μελετά τη χρήση σύγχρονων τεχνολογιών για τη δημιουργία
% ενός συστήματος Ενεργής Παρακολούθησης (Active Monitoring) σε συνδυασμό με μία SaaS εφαρμογή
% που θα παρουσιάζει μέσα από διαγράμματα τα αποτελέσματα της ανάλυσης της πληροφορίας που εξάγεται.

% Εξετάζονται διάφοροι τρόποι και υλοποιήσεις που δοκιμάστηκαν κατά τη διάρκεια
% εκπόνησης της διπλωματικής αυτής εργασίας. Ακόμα αναλύονται τα αποτελέσματα
% που παρήχθησαν καθόλη της διάρκεια των πειραμάτων που διενεργήθηκαν.  

Η παρούσα διπλωματική εργασία μελετά τη χρήση σύγχρονων τεχνολογιών για τη δημιουργία ενός συστήματος που κατά κύριο λόγο
θα αποτελεί έναν μηχανισμό Ενεργής Παρακολούθησης (Active Monitoring) σε συνδυασμό με συγκεκριμένες λειτουργίες 
Παθητικής Παρακολούθησης (Passive Monitoring), καθώς και μία SaaS (Software as a Service) εφαρμογή που θα παρουσιάζει μέσα από
διαγράμματα τα αποτελέσματα της ανάλυσης που κάνει το προαναφερθέν σύστημα. Αξίζει να αναφερθεί ότι για λόγους "γενίκευσης" το σύστημα Παθητικής Παρακολούθησης θα αποτελεί μία ξεχωριστή οντότητα που θα συλλέγει δεδομένα μόνο από εφαρμογές που χρησιμοποιούν τεχνολογία Docker για την ανάπτυξή/λειτουργία τους.

Στη συνέχεια θα εξεταστεί ο τρόπος δημιουργίας και υλοποίησης του συστήματος που δημιουργήσαμε, καθώς και τα αποτελέσματα και οι μετρήσεις που παρήχθησαν από τα πειράματα που διενεργήθηκαν.  