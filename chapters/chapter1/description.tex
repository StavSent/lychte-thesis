\section{Περιγραφή του Προβλήματος}
\label{section:problem_description}

Η Παρακολούθηση (Monitoring) ενός συστήματος που "ζει" στο χώρο του διαδικτύου μπορεί να γίνει κυρίως με δύο τρόπους:

\begin{itemize}
	\item \textbf{Ενεργή Παρακολούθηση (Active Monitoring)}: έχει περισσότερο προγνωστικό και προληπτικό χαρακτήρα.
		Συχνά αναφέρεται και ως \textbf{Συνθετική παρακολούθηση (Synthetic Monitoring)}, λόγω της φύσης των ενεργειών της.
		Ουσιαστικά δημιουργεί πλασματικά api calls και όχι πραγματικά δεδομένα χρηστών 
		προκειμένου να ελεγχθεί η απόκριση του υπό μελέτη συστήματος. Η συχνότητα αποστολής των
		συνθετικών αιτημάτων συνήθως ρυθμίζεται από το χρήστη.
	\item \textbf{Παθητική Παρακολούθηση (Passive Monitoring)}: παρέχει μία πιο πλήρη εικόνα σχετικά με πως χρησιμοποιούνται οι πόροι του δικτύου
		καταγράφοντας, αποθηκεύοντας και αναλύοντας τα δεδομένα του χρήστη. Για αυτό πολλές φορές αναφέρεται στη βιβλιογραφία ως \textbf{Παρακολούθηση Πραγματικών Χρηστών (Real User Monitoring - RUM)}. Έτσι μπορεί κανείς
		να εντοπίσει τις τάσεις χρήσης του δικτύου για τη βελτίωση και βελτιστοποίησή του συστήματος.
\end{itemize}


\begin{table}
	\begin{center}
		\caption{Χαρακτηριστικά Ενεργής και Παθητικής Παρακολούθησης}
		\label{tab:active_and_passive_monitoring}
		\begin{tabular}{ | c | c | }
			\hline
				\thead{Ενεργή Παρακολούθση \\ (Active Monitoring)} & \thead{Παθητική Παρακολούθηση \\ (Passive Monitoring)} \\
			\hline
				% \makecell{$\bullet$ Παράγει μικρή ποσότητα \\ δεδομένων} & \makecell{$\bullet$ Παράγει μεγάλη ποσότητα \\ δεδομένων} \\
				\makecell{$\bullet$ Στηρίζεται σε συνθετικά \\ API calls} & \makecell{$\bullet$ Αναλύει δεδομένα πραγματικών \\ χρηστών} \\
				\makecell{$\bullet$ Παράγει δεδομένα για συγκεκριμένες \\ πτυχές του δικτύου} & \makecell{$\bullet$ Πλήρης εικόνα της απόδοσης \\ του συστήματος} \\
				\makecell{$\bullet$ Μπορεί να μετρήσει την κίνηση \\ εντός και εκτός του δικτύου} & \makecell{$\bullet$ Μετράει κίνηση μόνο \\ εντός του δικτύου} \\
				\makecell{$\bullet$ Μπορεί να εντοπίσει προβλήματα \\ πριν ακόμα μπορέσουν να τα \\ εντοπίσουν οι χρήστες} & \makecell{$\bullet$ Εντοπίζει προβλήματα που \\ εμφανίζονται εκείνη τη στιγμή} \\
			\hline
		\end{tabular}
	\end{center}
\end{table}

Και οι δύο μέθοδοι στηρίζονται στη συλλογή δεδομένων, είτε εντός είτε εκτός του συστήματος που μελετάμε. Πιο συγκεκριμένα μπορούμε να δούμε τα χαρατηριστικά των δύο μεθόδων στον πίνακα \ref{tab:active_and_passive_monitoring}.

% Όπως είναι εμφανές
% η παθητική παρακολούθηση γίνεται πάνω στο σύστημα που θέλουμε να μελετήσουμε, πράγμα το οποίο σήμαινει ότι σαν εξωτερικοί παράγοντες στο
% σύστημα δεν θα μπορέσουμε να προσφέρουμε ανάλογες υπηρεσίες. Για το λόγο αυτό συνεχίζουμε την ανάλυση στο πλαίσιο της Ενεργής Παρακολούθησης Δικτύων.

Πιο συγκεκριμένα όσον αφορά την Ενεργή Παρακολούθηση, οι βασικοί λόγοι που κρίνεται επιτακτική η ανάγκη χρήσης τέτοιων υπηρεσιών όπως αναφέρεται και στα \cite{web_server_monitoring}, \cite{synthetic_monitoring_using_http_archive}
είναι οι εξής:

\begin{itemize}
	\item Βελτίωση προβλημάτων που σχετίζονται με την απόδοση του συστήματος
		πρωτού τα βιώσουν οι πραγματικοί χρήστες του συστήματος
	\item Ύπαρξη κάποιας μονάδας αξιολόγησης της απόδοσης του
	\item Αξιολόγηση του συστήματος υπό μεγαλύτερο φορτίο
	\item Διασφάλιση της Συμφωνίας Επιπέδου Υπηρεσιών (Service Level Agreement - SLA), μεταξύ
		του παρόχου υπηρεσιών και των χρηστών
	\item Παρέχει χρήσιμα δεδομένα ακόμα και σε καινούργια συστήματα που ακόμα μπορεί να μην έχουν χρήστες   
\end{itemize}

Σε ορισμένες περιπτώσεις μάλιστα, μπορούν να συνδυαστούν μέθοδοι Ενεργής και Παθητικής Παρακολούθησης, για να έχουμε μία πιο πλήρη εικόνα του υπό μελέτη συστήματος και να μπορούμε να κάνουμε αλλαγές αυτόματα πάνω στο ίδιο, για τη βελτίωση της απόδοσής του.


