\chapter{Εισαγωγή}
\label{chapter:intro}

Τα τελευταία χρόνια, ο κλάδος του Διαδικτύου προσεγγίζει ένα μεγαλύτερο κομμάτι ανθρώπων, τόσο από τη μεριά του καταναλωτή 
όσο και από τη μεριά του παραγωγού. Όσο αφορά τον καταναλωτή οι δυνατότητες που του προσφέρονται μπορούν να διακριθούν στους εξής τομέις:

\begin{itemize}
	\item Επικοινωνία: το διαδίκτυο παρέχει τη δυνατότητα άμεσης επικοινωνίας μεταξύ μεγάλων αποστάσεων, που δεν περιορίζεται μόνο στο ακουστικό
			ερέθισμα, αλλά επιτρέπει και την μετάδοση οπτικο-ακουστικής πληροφορίας
	\item Πρόσβαση Πληροφορίας: ίσως το σημεντικότερο αγαθό που προσφέρει το διαδίκτυο είναι η πληθώρα πληροφορίας
			που στεγάζει. Μηχανές Αναζήτηση (search engines), Online Βάσεις Δεδομένων (online databases),
			και άλλου είδους εφαρμογών εκπαιδευτικού χαρακτήρα που δίνουν πρόσβαση σε άτομα που το επιθυμούν, να κάνουν έρευνα
	\item Ποιότητα ζωής: σε αυτή την κατηγορία περιλαμβάνονται όλες εκείνες οι υπηρεσίες που
			διευκολύνουν την καθημερινότητα των χρηστών. Online αγορές (eshops) που γλιτώνουν την αναμονή σε ουρές ή ακόμα
			επιτρέπουν την εύκολη αγορά προϊόντων από απομακρυσμένες περιοχές του πλανήτη, ψυχαγωγία και πρόσβαση σε
			υπηρεσίες που επιταχύνουν ενέργειες που υπό άλλες περιπτώσεις θα ήταν χρονοβόρες (online banking,
			πληρωμή λογαριασμών, κρατήσεις ξενοδοχείων/εισητηρίων)
\end{itemize}

Από τη μεριά του παραγωγού, τα μέσα που υπάρχουν για την ανάπτυξη τέτοιων εφαρμογών/υπηρεσιών/συστημάτων
μέρα με τη μέρα αυξάνονται. Η ραγδαία εξέλιξη στον χώρο των cloud υποδομών, καθιστά ευκολότερη και επισπεύδει
τόσο την δημιουργία διαδικτυακών εφαρμογών, και σελιδών σε ένα γενικότερο πλαίσιο, όσο και την μεγέθυνση και αύξηση αυτών (scale up).
Μάλιστα η επιλογή κατάλληλου παρόχου τέτοιων υπηρεσιών αποτελεί ένα αρκετά σημεντικό αντικείμενο μελέτης \cite{cloud_service_provider_evaluation}.
Πέρα από τον οικομικό παράγοντα θα πρέπει να προσμετρηθούν οι παροχές, τα πλεονεκτήματα αλλά και η αποδοτικότητα που κάθε ένας προσφέρει.

Βλέποντας λοιπόν το πόσο συνυφασμένη είναι η ζωή του σύγχρονου ανθρώπου με το δίκτυο αλλά και τις δυνατότητες και τα μέσα
που έχει ο καθένας για να αναπτύξει εφαρμογές σε αυτό, καθίσταται επιτακτική η ανάγκη ύπαρξης μηχανισμών 
που θα αναγνωρίζουν σφάλματα (bugs) και θα επιβλέπουν την ορθή λειτουργία των υπό μελέτη συστημάτων καθόλη τη διάρκεια ζωής τους.

\section{Περιγραφή του Προβλήματος}
\label{section:problem_description}

Η Παρακολούθηση (Monitoring) ενός συστήματος που "ζει" στο χώρο του διαδικτύου μπορεί να γίνει κυρίως με δύο τρόπους:

\begin{itemize}
	\item \textbf{Ενεργή Παρακολούθηση (Active Monitoring)}: έχει περισσότερο προγνωστικό και προληπτικό χαρακτήρα.
		Συχνά αναφέρεται και ως \textbf{Συνθετική παρακολούθηση (Synthetic Monitoring)}, λόγω της φύσης των ενεργειών της.
		Ουσιαστικά δημιουργεί πλασματικά api calls και όχι πραγματικά δεδομένα χρηστών 
		προκειμένου να ελεγχθεί η απόκριση του υπό μελέτη συστήματος. Η συχνότητα αποστολής των
		συνθετικών αιτημάτων συνήθως ρυθμίζεται από το χρήστη.
	\item \textbf{Παθητική Παρακολούθηση (Passive Monitoring)}: παρέχει μία πιο πλήρη εικόνα σχετικά με πως χρησιμοποιούνται οι πόροι του δικτύου
		καταγράφοντας, αποθηκεύοντας και αναλύοντας τα δεδομένα του χρήστη. Για αυτό πολλές φορές αναφέρεται στη βιβλιογραφία ως \textbf{Παρακολούθηση Πραγματικών Χρηστών (Real User Monitoring - RUM)}. Έτσι μπορεί κανείς
		να εντοπίσει τις τάσεις χρήσης του δικτύου για τη βελτίωση και βελτιστοποίησή του συστήματος.
\end{itemize}


\begin{table}[H]
	\begin{center}
		\caption{Χαρακτηριστικά Ενεργής και Παθητικής Παρακολούθησης}
		\label{tab:active_vs_passive_monitoring}
		\begin{tabular}{ | c | c | }
			\hline
				\thead{Ενεργή Παρακολούθση \\ (Active Monitoring)} & \thead{Παθητική Παρακολούθηση \\ (Passive Monitoring)} \\
			\hline
				% \makecell{$\bullet$ Παράγει μικρή ποσότητα \\ δεδομένων} & \makecell{$\bullet$ Παράγει μεγάλη ποσότητα \\ δεδομένων} \\
				\makecell{$\bullet$ Στηρίζεται σε συνθετικά \\ API calls} & \makecell{$\bullet$ Αναλύει δεδομένα πραγματικών \\ χρηστών} \\
				\makecell{$\bullet$ Παράγει δεδομένα για συγκεκριμένες \\ πτυχές του δικτύου} & \makecell{$\bullet$ Πλήρης εικόνα της απόδοσης \\ του δικτύου} \\
				\makecell{$\bullet$ Μπορεί να μετρήσει την κίνηση \\ εντός και εκτός του δικτύου} & \makecell{$\bullet$ Μετράει κίνηση μόνο \\ εντός του δικτύου} \\
				\makecell{$\bullet$ Μπορεί να εντοπίσει προβλήματα \\ πριν ακόμα μπορέσουν να τα \\ εντοπίσουν οι χρήστες} & \makecell{$\bullet$ Εντοπίζει προβλήματα που \\ εμφανίζονται εκείνη τη στιγμή} \\
			\hline
		\end{tabular}
	\end{center}
\end{table}

Και οι δύο μέθοδοι έχουν πλεονεκτήματα και μειονεκτήματα, τα οποία φαίνονται και στον παραπάνω πίνακα \ref{tab:active_vs_passive_monitoring}. Όπως είναι εμφανές
η παθητική παρακολούθηση γίνεται πάνω στο σύστημα που θέλουμε να μελετήσουμε, πράγμα το οποίο σήμαινει ότι σαν εξωτερικοί παράγοντες στο
σύστημα δεν θα μπορέσουμε να προσφέρουμε ανάλογες υπηρεσίες. Για το λόγο αυτό συνεχίζουμε την ανάλυση στο πλαίσιο της Ενεργής Παρακολούθησης Δικτύων.

Οι βασικοί λόγοι που χρειάζονται τέτοιου είδους υπηρεσίες όπως αναφέρεται και στα \cite{web_server_monitoring}, \cite{synthetic_monitoring_using_http_archive}
είναι οι εξής:

\begin{itemize}
	\item Βελτίωση προβλημάτων που σχετίζονται με την απόδοση του συστήματος
		πρωτού τα βιώσουν οι πραγματικοί χρήστες του συστήματος
	\item Ύπαρξη κάποιας μονάδας αξιολόγησης της απόδοσης του
	\item Αξιολόγηση του συστήματος υπό μεγαλύτερο φορτίο
	\item Διασφάλιση της Συμφωνίας Επιπέδου Υπηρεσιών (Service Level Agreement - SLA), μεταξύ
		του παρόχου υπηρεσιών και των χρηστών
	\item Παρέχει χρήσιμα δεδομένα ακόμα και σε καινούργια συστήματα που ακόμα μπορεί να μην έχουν χρήστες   
\end{itemize}

\section{Σκοπός - Συνεισφορά της Διπλωματικής Εργασίας}
\label{section:contribution}

Η παρούσα διπλωματική εργασία μελετά τη χρήση σύγχρονων τεχνολογιών για τη δημιουργία
ενός συστήματος Ενεργής Παρακολούθησης (Active Monitoring) σε συνδυασμό με μία SaaS εφαρμογή
που θα παρουσιάζει μέσα από διαγράμματα τα αποτελέσματα της ανάλυσης της πληροφορίας που εξάγεται.

Εξετάζονται διάφοροι τρόποι και υλοποιήσεις που δοκιμάστηκαν κατά τη διάρκεια
εκπόνησεις της διπλωματικής αυτής εργασία, και τέλος θα αναλύσουμε τα αποτελέσματα
που παράξαμε καθόλη της διάρκεια των πειραμάτων που διενεργήθηκαν.  
\section{Διάρθρωση της Αναφοράς}
\label{section:layout}

Η διάρθρωση της παρούσας διπλωματικής εργασίας είναι η εξής:

\begin{itemize}
  \item{\textbf{Κεφάλαιο \ref{chapter:theory}:} 
		Περιγράφονται τα βασικά εργαλεία και θεωρητικά στοιχεία
		στα οποία βασίστηκαν οι υλοποιήσεις
    }
  \item{\textbf{Κεφάλαιο 3} Αναφορά συστημάτων που ήδη χρησιμοποιούνται	
		και παράθεση διαφορών με την υλοποίησή μας
    }
  \item{\textbf{Κεφάλαιο 4} Περιγραφή των υλοποιήσεων
  		και πλήρης περιγραφή του τελικού συστήματος
    }
  \item{\textbf{Κεφάλαιο 5} Παρουσιάζονται τα τελικά συμπεράσματα.
    }
  \item{\textbf{Κεφάλαιο 6} Προτείνονται θέματα για μελλοντική
      μελέτη, αλλαγές και επεκτάσεις.
    }
\end{itemize}

